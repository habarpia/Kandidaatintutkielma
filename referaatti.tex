Opiskelijoiden käyttö tietojenkäsittelytieteen opettamisessa Stanfordin yliopistossa



Tausta

Stuart Reges, John McGrory ja Jeff Smith kirjoittavat artikkelissaan ”The effective use of undergraduates to staff large introductory CS courses” (1988) mitä etua Stanfordin yliopistossa saatiin opiskelijoiden käyttämisestä alkeiskurssien opetuksessa. 
-CS198


Erilaiset kokeilut

Uusien opiskelijoiden hyödyntämiseen tehokkaasti opetuksessa kokeiltiin useita erilaisia rakenteita. Ensimmäisten kokeilujen ongelmana oli se, että tehtävien arvostelijat eivät tienneet tarpeeksi hyvin, mikä kurssilla oli tärkeää. Lisäksi joidenkin mielestä oli ongelmallista palkita opiskelijat opintopisteillä palkan sijaan. Lopulta päädyttiin malliin, jossa uudet opiskelijat saavat opintopisteitä ja vanhemmat palkkaa.



Opiskelijoiden käyttämisen edut

- tuntevat opetettavat opiskelijat, materiaalin ja tietokoneet, lähellä opiskelijoita
-tekninen osaaminen?



Opetus

- viikoittainen keskustelusessio
- yksilöllinen palaute viikoittain


Opettamisen oppiminen

Tärkeä osa CS198:a oli opiskelijoiden saama kokemus opettamisesta. Opiskelijat osallistuivat tapaamisiin, joissa keskusteltiin opetustavoista ja simuloitiin opetustilanteita. Kokeneemmat ryhmänohjaajat pystyivät jakamaan kokemustaan uusille. Tapaamissa myös pyrittiin jakamaan tietoa eri kurssien välillä.
