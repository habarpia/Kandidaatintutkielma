\documentclass[finnish]{tktltiki2}


\usepackage[utf8]{inputenc}
\usepackage{lmodern}
\usepackage{microtype}
\usepackage{amsfonts,amsmath,amssymb,amsthm,booktabs,color,enumitem,graphicx}
\usepackage[pdftex,hidelinks]{hyperref}

\makeatletter
\AtBeginDocument{\hypersetup{pdftitle = {\@title}, pdfauthor = {\@author}}}
\makeatother

\usepackage[fixlanguage]{babelbib}
\selectbiblanguage{finnish}

\usepackage[nottoc,numbib]{tocbibind}
\settocbibname{Lähteet}

\newtheorem{lau}{Lause}
\newtheorem{lem}[lau]{Lemma}
\newtheorem{kor}[lau]{Korollaari}

\theoremstyle{definition}
\newtheorem{maar}[lau]{Määritelmä}
\newtheorem{ong}{Ongelma}
\newtheorem{alg}[lau]{Algoritmi}
\newtheorem{esim}[lau]{Esimerkki}

\theoremstyle{remark}
\newtheorem*{huom}{Huomautus}

\title{Opiskelijoiden käyttö tietojenkäsittelytieteen opettamisessa Stanfordin yliopistossa}
\author{Hanna Arpiainen}
\date{\today}
\level{Referaatti}
\abstract{Referaatti.}

\keywords{avainsana 1, avainsana 2, avainsana 3}
\classification{}

\begin{document}

\maketitle
\makeabstract

\tableofcontents
\newpage



\section{Johdanto}

Stuart Reges, John McGory ja Jeff Smith tutkivat artikkelissaan ”The effective use of undergraduates to staff large introductory CS courses” (1988), miten voisi parantaa tietojenkäsittelytieteen alkeiskurssien opetuksen laatua ja määrää ilman, että kustannukset kohoaisivat. Stanfordin yliopistossa ongelma koetettiin ratkaista käyttämällä opiskelijoita alkeiskurssien opetukseen. Kokeilu oli onnistunut, ja opiskelijoiden käyttöä alkeiskurssien ohjaajina jatkettiin edelleen. 

\\

Uusien opiskelijoiden hyödyntämiseen opetuksessa kokeiltiin useita erilaisia rakenteita. Ensimmäisten kokeilujen ongelmana oli se, että  tehtävien arvostelijat eivät tienneet tarpeeksi hyvin, mikä kurssilla oli tärkeää. Lopulta päädyttiin malliin, jossa uudet opiskelijat saavat opintopisteitä ja vanhemmat palkkaa. Opettamisesta luotiin kurssi CS198 "Tietojenkäsittelytieteen opettaminen", jota pidettiin onnistuneena ja opettavaisena kokemuksena, vaikka opettamisen palkitseminen opintopisteillä jakoi mielipiteita.





\section{Ohjaajien työtehtävät}

Kukin CS198:n opiskelija toimi ohjaajana ryhmälliselle alkeiskurssin opiskelijoita. Ryhmänohjaajien tehtäviin kuului järjestää ryhmälleen viikoittainen keskustelusessio, jossa käytiin läpi kurssin sisältöä. Keskustelusessiot eivät noudata mitään tiukkaa kaavaa, vaan ryhmänohjaaja muokkaa niiden sisällön vastaamaan ryhmän tarpeita. Ryhmänohjaaja saattoi esimerkiksi kerrata luentojen asioita, esittää lisäesimerkkejä tai vastata opiskelijoiden esittämiin kysymyksiin. Mikäli ryhmä oli ymmärtänyt jonkin tärkeän asian väärin, ryhmänohjaaja saattoi joutua hylkäämään aiemmat suunnitelmansa korjatakseen tilanteen.


\\

Ryhmänohjaajat tapasivat kerran viikossa kaikki ryhmänsä opiskelijat yksilöllisessä palautekeskustelussa. Näissä keskusteluissa ryhmänohjaaja sai kuvan opiskelijan edistymisestä kurssilla. Ryhmänohjaajat myös päivystivät tietokoneluokassa valmiina auttamaan kurssin opiskelijoita kurssiin liittyvissä ongelmissa.





\section{Opettamisen oppiminen}

Tärkeä osa CS198:a oli opiskelijoiden saama kokemus opettamisesta. Suurin osa CS198:n opiskelijoiden kouluttamisesta oli käytännön harjoittelua, mutta he osallistuivat viikoittaisiin tapaamisiin ohjaajan ja muiden opiskelijoiden kanssa. Tapaamisissa pyrittiin opettamaan opettamista avoimien kesustelujen ja opetustilanteiden simuloinnin avulla pelkän luennoinnin sijaan. Kokeneempia ryhmänohjaajia rohkaistiin osallistumaan tapaamisiin, koska heidän kokemuksensa ja esimerkkinsä oli erittäin hyödyllinen uusille ryhmänohjaajille. Samalla he pystyivät myös päivittämään omia taitojaan. 

\\

Tapaamissa pyrittiin jakamaan ryhmänohjaajille tietoa eri kurssien sisällöistä ja tehtävistä, jotta he pystyisivät neuvomaan opiskelijoita mahdollisimman tehokkaasti. Tapaamisissa eri alkeiskurssien ryhmänohjaajat ja assistentit myös pystyivät tapaamaan ja tiedottamaan oman kurssinsa edistymisestä ja mahdollisista ongelmista.


\section{Opiskelijoiden käyttämisen edut}

Artikkelin mukaan opinnoissaan pidemmälle edenneiden opiskelijoiden korvaaminen uudemmilla opetustehtävissä saatiin hyviä tuloksia. Uudessa, kommunikointia korostavassa mallissa opiskelijoiden tarpeet pystytään huomioimaan paremmin, ja tieto liikkuu eri kurssien välillä. Uudessa mallissa yhden ohjaajan vastuulla oli vähemmän opiskelijoita, joten opiskelijat saivat enemmän yksilöllistä huomiota.

\\

Opintopisteillä palkittavat opiskelijat olivat halvempia kuin kokeneet, palkkaa saavat opiskelijat. Lisäksi he olivat parempia, sillä he tunsivat Stanfordin yliopiston alkeiskurssit ja tietokoneet hyvin ja tapasivat opettamiaan opiskelijoita säännöllisesti.




\bibliographystyle{babplain}
\bibliography{lahteet}


\end{document}
