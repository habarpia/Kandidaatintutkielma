\documentclass[finnish]{tktltiki2}


\usepackage[utf8]{inputenc}
\usepackage{lmodern}
\usepackage{microtype}
\usepackage{amsfonts,amsmath,amssymb,amsthm,booktabs,color,enumitem,graphicx}
\usepackage[pdftex,hidelinks]{hyperref}

\makeatletter
\AtBeginDocument{\hypersetup{pdftitle = {\@title}, pdfauthor = {\@author}}}
\makeatother

\usepackage[fixlanguage]{babelbib}
\selectbiblanguage{finnish}

\usepackage[nottoc,numbib]{tocbibind}
\settocbibname{Lähteet}

\newtheorem{lau}{Lause}
\newtheorem{lem}[lau]{Lemma}
\newtheorem{kor}[lau]{Korollaari}

\theoremstyle{definition}
\newtheorem{maar}[lau]{Määritelmä}
\newtheorem{ong}{Ongelma}
\newtheorem{alg}[lau]{Algoritmi}
\newtheorem{esim}[lau]{Esimerkki}

\theoremstyle{remark}
\newtheorem*{huom}{Huomautus}

\title{Opiskelijoiden hyödyntäminen tietojenkäsittelytieteen opettamisessa}
\author{Hanna Arpiainen}
\date{\today}
\level{tutkielma}
\abstract{tutkielma.}

\keywords{avainsana 1, avainsana 2, avainsana 3}
\classification{}

\begin{document}
\setcounter{page}{2}
\maketitle
\makeabstract

\tableofcontents
\newpage





\section{Johdanto}
Tietojenkäsittelytieteen opetuksen tarve on kasvanut. Ongelmana on usein taloudellisten resurssien puute, mikä johtaa opiskelupaikkojen lukumäärän rajoittamiseen. Lisäksi etenkin 1980-luvulla osaavista opettajista oli pulaa, joten henkilökuntaan jouduttiin palkkaamaan riittämättömästi koulutettuja ulkopuolisia \cite{Roberts11}. Teollisuuden menestys houkuttelee opiskelijoita siirtymään nopeasti työelämään opintojen jatkamisen sijaan, jolloin on yhä vaikeampi saada uusia opettajia uusille tietojenkäsittelytieteen opiskelijoille \cite{Roberts99}.
\\
Ongelmaa voi ratkaista kehittämällä yhteistyötä teollisuuden kanssa, koska sillä on taloudellisia resursseja tukea opetusta \cite{Roberts11}. Toinen ratkaisu on palkata alkeiskursseille mahdollisimman pätevää henkilökuntaa. 
\\
Kolmas ratkaisu ongelmaan on hyödyntää opiskelijoita opetuksessa, mihin tässä tutkielmassa keskitytään. Monet yliopistot ovat päättäneet hyödyntää opiskelijoita kurssien opetukseen, sillä he ovat halvempia kuin pitkälle koulutettu työvoima. Taloudellisen edun lisäksi opiskelijoiden on havaittu olevan hyödyllisiä monella muullakin tavalla, sillä opiskelijoiden käytön ohjaajina on havaittu esimerkiksi nostavan opetuksen laatua ja parantavan kurssin ilmapiiriä.
\\
Vaikka opiskelijoita on hyödynnetty tietojenkäsittelytieteen opetuksessa jo pitkään, aihetta on tutkittu niukasti. Opiskelijoiden käyttämistä ohjaajina käsittelevät artikkelit antavat usein yksipuolisen myönteisen kuvan aiheesta, koska mahdollisista haitoista tai huonosti toimivan toteutuksen riskeistä on kirjoitettu vain vähän. 


\section{Muutokset tietojenkäsittelytieteen opetuksessa}

\subsection{Opiskelijamäärien kasvu}
Tietojenkäsittelytieteen opiskelijoiden lukumäärän vaihtelu on syklistä \cite{Roberts11}. Esimerkiksi kotitietokoneiden ja World Wide Webin yleistyminen kasvattivat alan suosiota ja opiskelijoiden määrä kasvoi, kun taas IT-kuplan puhkeaminen 2000-luvun alussa vähensi opiskelijamäärää. Ajoittaisista taantumisista huolimatta opiskelijoiden määrä on pitkällä ajanjaksolla kasvanut, ja teollisuus tarvitsee yhä enemmän osaajia. Muiden aineiden opiskelijat haluavat opiskella tietojenkäsittelytieteen perusteita, koska he uskovat, että ohjelmointitaidot auttavat heitä työmarkkinoilla. Tämä lisää tietojenkäsittelytieteen opetuksen tarvetta. Etenkin alkeiskursseille pitäisi siis pystyä järjestämään riittävästi opetusta. 

\subsection{Pula opettajista}
Tietojenkäsittelytieteen alaa on vaivannut puute pätevistä opettajista \cite{Roberts99}. Teollisuuden menestys houkuttelee opiskelijat jättämään opintonsa ja siirtymään töihin, jolloin on yhä vaikeampi kouluttaa uusia opettajia. Työ\-voi\-ma\-pu\-la aiheuttaa ylitöitä ja stressiä opettajille, mikä lisää heidän ovat halukkuuttaan siirtyä muihin töihin. Opettajien puute vaikeuttaa kasvaviin opiskelijamääriin vastaamista.

\subsection{Suurten luentojen ongelmat}

Kurssikoon kasvaessa kasvaa myös kurssin tehtävien korjaukseen ja opiskelijoiden henkilökohtaiseen ohjaukseen tarvittava työmäärä \cite{Kay98}. Suurissa opiskelijamäärissä voi parhaiden ja heikoimpien opiskelijoiden ero olla hyvin suuri, joten ääripäät tarvitsisivat erityisen paljon yksilöllistä huomiota. Kokeiden ja tehtävien tekeminen vaikeutuu oppilasmäärän kasvaessa, jos tehtävistä pitää saada mahdollisimman nopeasti tarkistettavia. Tehtävät tulisi myös olla määritellä mahdollisimman selkeästi, sillä epäselvien kysymysten aiheuttamia väärinymmärryksiä tulee suurten opiskelijamäärien kanssa enemmän. Suurella kurssilla voi olla haasteellista taata, että kaikki opiskelijat arvostellaan tasavertaisesti \cite{Chamillard02}. Kurssikoon kasvaessa vaikeutuu myös sen hallinnointi, kuten kurssinateriaalin valmistelu ja jakaminen opiskelijoille, neuvonnan tarjoaminen tehtävissä apua tarvitseville opiskelijoille ja henkilökunnan hallinnointi. 
\\
Suurten ryhmäkokojen seurauksena opiskelijat uhkaavat hukkua massaan ja voivat sen seurauksena menettää mielenkiintonsa alaa kohti \cite{Kay98}. Pelkkä luentojen passiivinen kuuntelu ei rohkaise opiskelijoita osallistumaan \cite{Kopp00}. Uusien opiskelijoiden voi olla hankala lähestyä kokenutta luennoijaa, ja luentoihin painottuvilla kursseilla voi olla vaikeuksia tarjota opiskelijoille roolimalleja. Monilla opiskelijoilla ei ole tietojenkäsittelytieteestä aiempaa kokemusta, ja suuret luennot ja yksin tehtävät harjoitukset voivat tuottaa heille hankaluuksia \cite{Murphy11}. Passiivinen luentojen kuuntelu ei vastaa tietojenkäsittelytieteen alan työelämää, jossa tarvitaan ongelmanratkaisutaitoja ja kykyä toimia ryhmässä \cite{Vihavainen}.
\\
Välttääkseen syrjäytymisen tunnetta opiskelijan olisi tärkeää päästä opiskeluun mukaan heti opintojen alussa \cite{Settle12}. Yliopisto-opintojen itsenäisyys voi olla uutta monille opiskelijoille, ja heidän voi olla helppo oppia huonoja opiskelutapoja ja olla välittämättä läsnäolon tärkeydestä kursseilla.
\\
Suurten luentojen passiivisella kuuntelemisella ja asioiden ulkoa opettelulla opiskelija voi yltää pintapuoliseen osaamiseen, mutta syvempään osaamiseen vaaditaan aktiivista opiskelua \cite{Boyer07}. Aktiivinen osallistuminen ehkäisee opintojen kesken jättämistä. Opiskelijoilla voi olla vaikeuksia keskittyä ja seurata opetusta suurilla luennoilla. Suurten luentokurssien opiskelijoilla voi olla heikompi itseluottamus omien kykyjensä, opintomenestyksensä tai tietojenkäsittelytieteen hyödyllisyyden tulevaisuudessa suhteen kuin pienryhmissä opiskelevilla.


\section{Oppilaat opettajina}
Kasvaviin opiskelijamääriin, suuriin luentokursseihin ja työvoimapulaan liittyviä ongelmia voi ratkaista käyttämällä opiskelijoita tietojenkäsittelytieteen opetuksessa. Opiskelijat voivat toimia ohjaajana ryhmälle alkeiskurssin o\-pis\-ke\-li\-joi\-ta \cite{Reges88}. Ohjaajat tarjoavat lisäopetusta kurssin varsinaisten luentojen tueksi ja auttavat kurssin opiskelijoita tehtävien kanssa. Ohjaajilla ei kuitenkaan välttämättä tarvitse olla omaa vastuuryhmäänsä, vaan he voivat antaa yleistä neuvontaa tietokoneluokissa \cite{Vikberg, Vihavainen}. Joissakin laitoksissa opiskelijat käyttävät enemmän aikaa ohjaajien kuin luennoijan kanssa \cite{Patitsas12_3}.
\\
Ryhmänohjaajien lisäksi opiskelijoita voi käyttää opetuksessa vertaistukena \cite{Tashakkori05}. Vertaistukea voi saada opiskelija, joka tarvitsee lisää tukea opintoihinsa, tai joka tarvitsee ylimääräisiä haasteita. Vertaismentorin tulee osata opettamansa  kurssin sisältö hyvin. Mentori tapaa neuvottavaansa ja avustaa tehtävien kanssa. Mentorointi voi mennä myös ristiin; opiskelija voi olla mentorina toiselle yhdellä kurssilla ja neuvottavansa mentoroitavana jollakin toisella.
\\
Opiskelijoiden käyttäminen ohjaajina vaatii laitokselta sitoutumista \cite{Kopp00}. Koska ohjaajien käyttämiseen tarvitaan ennakkoinvestionteja, kannattaa laitoksen järjestää ohjaajia vain kursseille, joita siellä on aiemmin opetettu. 
\\
Joissakin laitoksissa voi ohjaajana toimiminen olla pakollinen osa tutkintoa \cite{Kay95}.


\subsection{Ohjaajien työtehtäviä}

Ryhmänohjaajien tehtäviin kuuluu esimerkiksi viikottaisten keskustelusessioiden pitäminen \cite{Reges88}. Keskustelusessiossa käydään läpi kurssin sisältöä, mutta sen ei tarvitse noudattaa mitään tiukkaa kaavaa, vaan ryhmänohjaaja voi muokata sen vastaamaan ryhmänsä tarpeita. Ryhmänohjaaja voi esimerkiksi kerrata luentojen asioita, esittää lisäesimerkkejä tai vastata opiskelijoiden esittämiin kysymyksiin. Viikkotapaamisten ei tarvitse olla vain ohjaajan esiintymistä, vaan ne voivat sisältää opiskelijoiden ryhmäkeskusteluja \cite{Decker06}. Tarvittaessa ryhmänohjaajan on oltava valmis hylkäämään aiemmat suunnitelmansa, mikäli selviää, että ryhmä on ymmärtänyt jonkin tärkeän asian väärin ja tilanne on korjattava nopeasti \cite{Reges88}.
\\
Ryhmänohjaajan tehtäviin voi kuulua viikottainen yksilöllinen palautekeskustelu jokaisen ryhmänsä opiskelijan kanssa. Opiskelija voi selittää ratkaisunsa johonkin tehtävään, tai kertoa, että ei ole ymmärtänyt jotakin kurssin asiaa. Ohjaaja taas voi selittää, mistä jokin virhe johtuu, ja miten sen voi jatkossa välttää. Näissä keskusteluissa ryhmänohjaaja saa kuvan opiskelijan edistymisestä kurssilla \cite{Reges88,Reges03}. Opiskelijan on tärkeä saada mielekästä palautetta edistymisestään \cite{Kurhila11}. Ohjaajalla voi olla vastaanottoaika, jonka aikana opiskelijat voivat käydä kysymässä häneltä yksilöllisiä vastauksia kurssiin liittyviin tehtäviin \cite{Decker06}.
\\
Ryhmänohjaajat voivat päivystää tietokoneluokassa valmiina auttamaan kurssin opiskelijoita kurssiin liittyvissä ongelmissa. Ruuhka-ajoille päivystäviä ohjaajia pyritään järjestämään enemmän \cite{Reges88, Reges03}, ja ohjaajat saattavat voida kutsua lisäapua, mikäli tietokoneluokka ruuhkautuu pahasti \cite{Kurhila11}. Ohjaajan ei tulisi antaa suoraa vastausta tehtävään, vaan ohjata ja neuvoa opetettavaa löytämään ratkaisu itse \cite{Vikberg, Kurhila11}. Ohjelmointitehtävien yhteydessä ohjaaja voi kannustaa opetettavaa kirjoittamaan hyvää ja luettavaa koodia esimerkiksi nimeämällä muuttujat selkeästi. Ohjaajat voivat erottuakseen joukosta täydessä tietokoneluokassa käyttää esimerkiksi turvaliivejä \cite{Vihavainen}. 
\\
Ohjaajat voivat pitää ennen koetta kertaustilaisuuden kurssin sisällöstä. Kertaustilaisuudessa he voivat kertoa, mitkä kurssin asiat ovat tärkeitä ja mitä kokeessa todennäköisesti kysytään, tai esitellä vanhoja kokeita ja niiden ratkaisuja \cite{Decker06}.
\\
Usein ohjaajan tehtäviin kuuluu tehtävien tarkistaminen \cite{Dickson11}. Kun ohjaajat osallistuvat tehtävien tarkastamiseen, voi tehtäviä olla enemmän, ja luennoija voi käyttää aikansa tehokkaammin esimerkiksi kirjoittamalla kommentteja tehtäviin. Ohjaajat voivat käydä kurssin materiaalin läpi ennen, kuin luennoija esittää sen opiskelijoille, ja huomauttaa mahdollisista virheistä tai puutteista \cite{Vikberg}. Ohjaajat voivat ratkaista kurssin tehtävät ennen kuin ne annetaan opiskelijoille, mikä auttaa heitä valmistautumaan neuvontaan \cite{Vihavainen}.
\\
Ohjaajan voidaan edellyttää osallistuvan kurssin luennoille pysyäkseen mahdollisimman hyvin perillä kurssin etenemisestä ja käsitellyistä esimerkeistä \cite{Reges03, Decker06}. Ohjaaja voi päivittää kurssin verkkosivuja \cite{Dickson11}, mikä vähentää luennoijan työtaakkaa ja auttaa opiskelijoita, joilla on vaikeuksia omien muistiinpanojen tekemisen kanssa. Luennoille osallistuva ohjaaja voi päivittää esimerkkejä verkkosivuille reaaliajassa. Ohjaajat voivat valvoa kurssikokeita \cite{Richards00}.



\subsection{Millainen on hyvä ohjaaja?}
On vaikea määritellä tarkkaan, millainen on hyvä ohjaaja, koska niin monet erilaiset ohjaajat ovat onnistuneet tehtävässään. Ohjaajalta vaaditaan kuitenkin aitoa innostusta auttaa muita oppimaan ja vastuullisuutta noudattaa kurssin käytäntöjä. Ohjaajan ei ole välttämätöntä olla ohjelmoinnin erityisosaaja, mutta hänen edellytetään ymmärtävän kurssin sisältö ja olevan valmis kehittämään itseään pystyäkseen auttamaan opetettaviaan riittävästi \cite{Reges88}. Ohjaaja voi jopa hyötyä siitä, että hänellä on itsellään ollut oppimisvaikeuksia, sillä se voi auttaa häntä ymmärtämään paremmin opetettaviensa vaikeuksia samojen tehtävien kanssa \cite{Decker06}.
\\
On havaittu, että uudet opiskelijat ovat yleensä parempia ohjaajia kuin vanhemmat, opinnoissaan pidemmälle edenneet opiskelijat \cite{Dickson11}. Uusilla ohjaajilla on muistissaan hyvä kuva kurssin pääasioista, koska he ovat itse käyneet kurssin vasta vähän aikaa sitten. Pitkälle edenneiden opiskelijoiden taas saattaa olla vaikeaa ymmärtää kurssin materiaalia, opetustekniikkaa tai alkeiskurssien opiskelijoiden ongelmia, ja heidän voi olla vaikeampi ymmärtää laitoksen tietokoneita \cite{Reges88}. Korkealla tutkintoarvolla ei välttämättä ole paljoa yhteistä opetuksen kanssa, ja materiaalin osaaminen on vain yksi osa opetuksesta, joten pitkälle edenneet opinnot eivät takaa menestystä opettajana \cite{Baldwin00}. Uusi ohjaaja oppii nopeasti, millainen on hyvä ohjaaja, mikäli hän on juuri itse ollut ohjattavana \cite{Vihavainen, Vikberg}. Ohjattavien voi olla vaikea suhtautua ohjaajaan, jos tämä on heitä kovin paljon vanhempi \cite{Decker06}.
\\
Tohtorin tai maisterin tutkinnon suorittanutta henkilökuntaa voi pitää uusia ja kouluttamattomia opiskelijoita parempana vaihtoehtona ohjaajaksi tai opettajaksi \cite{Baldwin00}. Kokenut opettaja voi helpommin viitata eri kurssien sisältöihin ja kertoa, miten kurssit liittyvät toisiinsa. Näin hän voi auttaa opiskelijoita näkemään materiaalin laajassa mittakaavassa eikä vain kokelmana yksittäisiä ja erillisiä kursseja. Laajat pohjatiedot voivat auttaa opettajaa tunnistamaan ja omaksumaan nopeasti pääkohdat materiaalista, jota hän ei tunne ennalta. Pitkälle koulutettu henkilökunta voi myös toimia roolimallina opiskelijoille, sillä tutkinton suorittaminen viittaa kykyyn sitoutua tutkimukseen ja ajatella luovasti ja itsenäisesti. Kokenut opettaja kannustaa opiskelijoita jatkamaan opinnoissaan ja ajattelemaan tietojenkäsittelytiedettä kiinnostavaana tieteenalana pelkän teollisuuteen siirtymisen sijaan. Akateemisen uran valitsevat opiskelijat tarvitsevat opinnoista asiantuntevaa neuvontaa, jota osaa parhaiten antaa ohjaaja, jolla on kokemusta aiheesta. Tämä auttaa opettajapulaan. Toisaalta opinnoissaan pitkälle edenneitä opiskelijoita voi olla vaikea houkutella ohjaajiksi, koska tietojenkäsittelytieteen alalla on tarjolla paljon muita työmahdollisuuksia \cite{Kay98}. 
\\
Aito kiinnostus opetettavien auttamiseen on keskeistä ohjaajan menestykselle \cite{Richards00}. Ohjaaja, joka on töissä vain palkan tai velvollisuuden vuoksi ja pyrkii pelkkään minimisuoritukseen, ei tue oppimista. Huono ohjaaja ei valmistaudu opetustilaisuuteen riittävästi tai vähättelee opetettaviaan tai opetettavaa materiaalia. Opiskelijat tuntevat tulevansa kohdelluksi epäoikeudenmukaisesti, jos he tietävät, että muilla ryhmillä on parempi ohjaaja kuin heillä. Aluksi heikolta ohjaajalta vaikuttava ehdokas voi osoittautua hyväksi ohjaajaksi, jos hänellä on innostusta kehittyä työssään tai hän saa enemmän itseluottamusta kurssin edetessä \cite{Dickson11}.
\\
Ohjaajan voi edellyttää olevan aktiivinen, vaikka opetettavat eivät itse kysyisikään mitään \cite{Vikberg}. Ohjaajan tulisi osata huomioida erilaisista taustoista tulevat opiskelijat \cite{Kay98}. Jos ohjaajalla on vastuullaan jokin tietty ryhmä opiskelijoita, hänen tulisi opetella heidän nimensä ja tutustuttava heihin \cite{Bernstein}.
\\
Pareittain työskenteleminen parantaa ohjaajien suoritusta \cite{Patitsas12_3}. Ohjaaja sitoutuu tehtäväänsä paremmin, jos hänellä on pari, joka odottaa, että työtehtävät jaetaan tasavertaisesti. Työparin läsnäolo luo turvallisuuden tunnetta, jos ohjaaja voi luottaa siihen, että pari auttaa ja tukee tarvittaessa. Pareittain työskentely parantaa ohjaajien viihtymistä työssään.
\\
Ohjaajan olisi osattava opetuskieli hyvin \cite{Richards00}. Kielivaikeuksista seuraa kom\-mu\-ni\-kaa\-tio-on\-gel\-mia, minkä takia opiskelijat eivät välttämättä ymmärrä ohjaajaa.


\subsection{Ohjaajien koulutus}

Uusien ohjaajien kouluttamiseen voi kuulua erillinen tietojenkäsittelytieteen opettamiseen keskittyvä kurssi \cite{Reges88, Roberts95} tai aloitusseminaari \cite{Sperry08}. Ohjaajilla voi olla paljon koulutusta opettamansa kurssin alussa, ja koulutustapaamiset vähenevät ja lopulta loppuvat kokonaan kurssin dedetssä \cite{Roberts95}. Myös teollisuuden edustajat voivat olla kiinnostuneita kouluttamaan ohjaajia \cite{Morgan02}. Joskus ohjaajilla ei ole mitään virallista koulutusta, vaan he oppivat työtä tekemällä ja luennoijan ja opiskelijoiden palautteen avulla \cite{Shannon98, Vihavainen}. Vaikka ohjaamisesta ei olisi varsinaista koulutusta, ohjaajana toimimisen palkitseminen opintopisteillä antaa opiskelijoille kuvan siitä, että kokemuksesta on tarkoitus oppia ja että ohjaajakurssille saa osallistua \cite{Vikberg}.
\\
Ohjaajien kouluttamiseen liittyy usein keskustelutilaisuuksia. Niissä voidaan esimerkiksi harjoitella tehtävien pisteyttämistä, vaikeiden käsitteiden opettamista tai hankalien opiskelijoiden käsittelyä \cite{Reges03}. Koulutuksessa voidaan käydä läpi erilaisia oppimistyylejä ja keskustella, millainen on hyvä tai huono ohjaaja \cite{Kay95}. Uudet ohjaajat voivat pitää harjoitusesitelmiä toisilleen, mistä he saavat esiintymiskokemusta ja palautetta. Useita erilaisia esitelmiä nähdessään ohjaajat voivat oppia uusia lähestymistapoja aiheeseen. Ohjaajat voivat harjoitella tehtävien ja kokeiden tekemistä. Ohjaajille voidaan antaa tehtäväksi keskustella, millaisen tie\-to\-jen\-kä\-sit\-te\-ly\-tie\-teen alkeiskurssin he suunnittelisivat. Kurssin suunnitteleminen itse voi saada ohjaajat ajattelemaan tietojenkäsittelytieteen opetukseen liittyviä laajoja kysymyksiä, kuten mitä sivuaineopiskelijoiden tulisi alasta oppia, tai millaiset pohjatiedot uusilla opiskelijoilla voi olettaa tietojenkäsittelytieteestä olevan. Opiskellessaan opetusta ryhmänä ohjaajat oppivat tuntemaan toisensa ja voivat ystävystyä \cite{Roberts95}.
\\
Kokeneet ohjaajat voivat osallistua uusien ohjaajien koulutukseen kerratakseen omia taitojaan, ja samalla he voivat jakaa kokemustaan ja toimia roolimalleina uusille ohjaajille \cite{Reges88}. Kokeneiden ohjaajien vertaisopetus ja rohkaisu nopeuttaa uusien ohjaajien koulutusta \cite{Decker06}. 
\\
Aina muodollista koulutusta ei juuri tarvita tai se ei ole mahdollista, jos uusia ohjaajia hankitaan jatkuvasti opetuksen edetessä. Kokeneet ohjaajat kouluttavat epävirallisesti uusia ohjaajia neuvomalla heitä \cite{Kurhila11}. 
\\
Kehittyäkseen työssään ohjaajien on tärkeää saada palautetta \cite{Patitsas12, Patitsas12_3}. Myönteinen palaute parantaa ohjaajan itseluottamusta, mikä puolestaan parantaa hänen opetustaitojaan.
\\
Vanhat ohjaajat voivat antaa hyödyllistä palautetta ohjaajien koulutuksesta \cite{Decker06}. Ohjaajat voivat suositella opettamiaan opiskelijoita uusiksi ohjaajiksi ja ehdottaa, millaisilla valintaperusteilla saadaan valittua parhaat ohjaajaehdokkaat. 



\subsection{Käytännön organisaatio}

Ohjaajia hyödyntävien kurssien henkilökunnan rakenne vaihtelee kursseittain ja yliopistoittain. Esimerkiksi pienessä yliopistossa voi toimia malli, jossa luennoijan lisäksi kurssin henkilökuntaan kuuluu vain muutama ohjaaja \cite{Dickson11}, kun taas suurissa laitoksissa kursseilla, joilla on paljon opiskelijoita, on hyödyllistä, että kurssin henkilökuntaan kuuluu ohjaajien työstä vastaava assistentti \cite{Reges03}. Assistentti toimii ohjaajien ja luennoijan välisenä kontaktina ja jakaa kurssin materiaalin eteenpäin ohjaajille. Hän voi auttaa luennoijaa kurssin sisällön kanssa. Assistentti voi olla joku kurssin ohjaajista tai erillinen tehtävään palkattu henkilö.
\\
Mikäli henkilökuntaa tarvitaan kursseille paljon, ohjaajien käyttäminen tarvitsee usein toimiakseen jonkinlaisen koordinaattorin. Koordinaattori voi olla esimerkiksi opinnoissaan pitkälle edennyt opiskelija, joka on aiemmin toiminut ryhmänohjaajana. Jos ohjaajia on paljon, voidaan tarvita useampia koordinaattoreita \cite{Roberts95}.
\\
Koordinaattorin tehtäviin kuuluu kurssin käytännön hallinnointi, eli esimerkiksi luokkahuoneiden varaaminen ja tarvittavien ohjaajien jakaminen kursseille. He myös hoitavat ohjaajien valinnan, palkkaamisen ja kouluttamisen \cite{Reges88,Roberts95}. Koordinaattori voi huomauttaa ohjaajan puutteellisesta toiminnasta \cite{Reges88}.
\\
Koordinaattorien kuuluu ylläpitää tiedon kulkua ja kommunikointia henkilökunnan välillä. Heidän tulee myös järjestää tarvittavat kommunikointikanavat opiskelijoiden ja ohjaajien välille \cite{Reges88}. Koordinaattorien tehtäviin kuuluu järjestää tapaamisia henkilökunnalle, jotta luennoija saa kuvan opiskelijoiden etenemisestä. Viikkopalaverissa ohjaajat voivat kertoa mahdollisista ongelmista, ja luennoija voi tarpeen tullen hidastaa opetustahtia tai selittää jonkin epäselväksi jääneen asian uudestaan paremmin. Palaverissa ohjaajat pääsevät tapaamaan toisiaan ja jakamaan tietoa. 
\\
Koordinaattorien kuuluu myös järjestää tapaamisia, joissa käydään läpi opetukseen liittyviä asioita \cite{Reges88, Roberts95}. Koordinaattori voi kerätä opiskelijoilta palautetta ohjaustilaisuuksista, ja esitellä palautteen ohjaajille henkilökunnan tapaamisessa \cite{Patitsas12_2}. Palautteen avulla koordinaattori voi suunnitella parannuksia seuraavalle lukukaudelle. Ohjaajan voidaan edellyttää kirjoittavan kurssin koordinaattorille työstään viikottainen raportti, jossa hän voi kertoa, mikä viikon opetuksessa meni hyvin ja mikä huonosti, ja miten kurssia voidaan parantaa \cite{Hug11}.








\section{Seurauksia}

\subsection{Hyötyjä}

Opiskelijoiden käyttämisestä ohjaajina on taloudellista etua yliopistolle, sillä heille maksettava palkka on yleensä matalampi kuin kokeneemmille assistenteille, tai heidät palkitaan opintopisteillä \cite{Reges88}.
\\
Uusien opiskelijoiden käyttäminen ohjaajina on parantanut opiskelijoiden arvosanoja tietojenkäsittelytieteen alkeiskursseilla \cite{Decker06, Kurhila11}. Kasvattamalla ohjaajien määrää voi henkilökohtaisen ohjauksen määrää lisätä ja passiivisia luentoja vähentää tai poistaa kokonaan \cite{Kurhila11}.
\\
Yliopistolla työskentelevät ohjaajat voivat toimia roolimalleina muille opiskelijoille \cite{Roberts95, Tashakkori05}, sillä opiskelijat tietävät, että ohjaajaksi pääsemiseen vaaditaan kelvollista menestymistä opinnoissa \cite{Bernstein}. Opiskelijoiden voi olla vaikea nähdä itsensä esimerkiksi kokeneena luennoijana, kun taas ohjaaja tarjoaa helpommin lähestyttävämmän roolimallin \cite{Roberts02}. Jos ohjaajat valitaan opiskelijoiden keskuudesta, voidaan henkilökunta saada paremmin vastaamaan opiskelijoita. Etenkin naisille ja vähemmistöjen edustajille voi olla tärkeää saada kaltaisensa roolimalli, jotta heidän on helpompi tuntea kuuluvansa joukkoon \cite{Morgan02}. Vähemmistöopiskelijat voivat jäädä opiskelun ulkopuolisten epämuodollisten ryhmien ulkopuolelle. Ryhmänohjaajien käyttäminen mahdollistaa paremmin pienryhmissä opiskelun, joka auttaa opiskelijoita tutustumaan toisiinsa. Yhteenkuuluvuuden tunne saa opiskelijat pysymään alalla paremmin.
\\
Ryhmänohjaajien käyttäminen voi toimia koulutuksena ja innoituksena uusille luennoijoille yliopiston tulevaisuudessa \cite{Roberts95, Morgan02}. Ryhmänohjaajia ohjaavat opettajat saavat tilaisuuden vaikuttaa ohjaajien näkemykseen opetusalasta \cite{Morgan02}. 
\\
Koska opiskelijoiden palkkaaminen ohjaajiksi on halvempaa kuin muiden vaihtoehtojen, voi heitä palkata enemmän. Näin opetettavat saavat enemmän yksilöllistä huomiota. Kurssilla voi olla enemmän pieniä tehtäviä, jotka innostavat opiskelijoita paremmin lukemaan materiaalia, kun tehtävien tarkastamiseen on käytössä enemmän henkilökuntaa \cite{Dickson11}. Kun ohjaajia on luokassa useita, opiskelijat saavat huomiota nopeammin \cite{Ferstl10}. Tehtävien suuri määrä vaikuttaa myös oppimiseen, sillä opiskelija oppii materiaalin sitä paremmin, mitä enemmän hän pääsee harjoittelemaan sitä käytännön tehtävissä \cite{Vikberg}. Jonkun tietyn henkilökunnan edustajan mahdollinen puolueellisuus ei välttämättä pääse vaikuttamaan opiskelijoihin, jos luokassa on samaan aikaan useita erilaisia ohjaajia \cite{Morgan02}.
\\
Kurssin ilmapiiriä rentouttaa, jos opiskelijat näkevät luennoijan ja ohjaajan tulevan hyvin toimeen keskenään \cite{Dickson11}. Mikäli ohjaaja on läsnä esimerkiksi luennolla ja esittää luennotsijalle kysymyksiä aiheesta, rohkaisee se opiskelijoitakin osallistumaan enemmän luennoilla. Kurssin opiskelijat hyötyvät, jos heidän keskuudessaan on ohjaaja, joka osaa heti kurssin alussa kertoa luennoijan opetustyylistä ja painotuksista. Motivoitunut ohjaaja motivoi opetettaviakin työskentelemään ahkerammin. Kurssin opiskelijoiden voi olla helpompi antaa palautetta ohjaajalle kuin luennoijalle \cite{Morgan02}.
\\
Ohjaajat pystyvät antamaan hyödyllistä palautetta kurssin toiminnasta ja antaa parannusehdotuksia \cite{Decker06} tai huomauttaa, jos materiaalissa on hänen mielestään puutteita \cite{Dickson11}. Ohjaaja voi olla jonkin tietojenkäsittelyn osa-alueen erityisosaaja ja tuoda osaamisensa kurssin hyödyksi. Aiheesta aidosti kiinnostuneilla opiskelijoilla voi olla luennoijaa enemmän aikaa omistautua kurssille. Kun luennoija esittelee kurssin sisällön ohjaajille, hän saa tilaisuuden käydä opetettava materiaali läpi ennen sen esittämistä luennolla \cite{Kopp00}. Ohjaajia hyödyntämällä kurssia voi kehittää ketterästi, kun ohjaajat voivat päivittää kurssimateriaalia silloin, kun luennoijalla ei ole aikaa siihen tai jos hän ei ole huomannut sen olevan jäljessä \cite{Dickson11}.
\\
Ohjaajat saavat palkkaa tai opintopisteitä, minkä lisäksi opettaminen parantaa heidän ryhmätyö- ja esiintymistaitojaan. Ohjaajat oppivat keskustelemaan tietokonetaidoiltaan eritasoisten opiskelijoiden kanssa \cite{Vikberg}. Opettaessaan he oppivat itsekin materiaalin paremmin \cite{Reges03}, koska jonkin asian osaa kunnolla vasta sitten, kun osaa opettaa sen toiselle \cite{Bernstein, Dickson11}. Opetettavien kannustaminen hyvään ohjelmointityyliin parantaa ohjaajankin ohjelmointityyliä \cite{Roberts95}. Ohjaajana toimiminen opettaa vastuukantoa ja liittää opiskelijan paremmin mukaan yliopiston yhteisöön \cite{Dickson11}. Koska ohjaajat yleensä ovat tekemisissä luennoijan kanssa tavallisia opiskelijoita enemmän, heillä on tilaisuus saada ylimääräistä ohjausta opintoihinsa \cite{Aminzadeh}. Ohjaajana toimiminen opettaa ajan hallintaa. Opettamalla muita ohjaaja voi oppia omista opiskelutottumuksistaan \cite{Ferstl10}. 
\\
Opettaessaan muita ohjaaja oppii paremmin ymmärtämään tarpeen jatkuvalle elinikäiselle opiskelulle \cite{Paxton05}. Ohjaajana toimimisesta voi saada hyvän mielen päästessään auttamaan muita \cite{Aminzadeh} ja saadessaan nähdä opetettaviensa edistyvän \cite{Kurhila11}. Ohjaajat voivat saada toisistaan seuraa ja osallistua keskenään erilaisiin vapaa-ajan aktiviteetteihin \cite{Roberts95}. Kokemus ohjaajana toimimisesta näyttää hyvältä ansioluettelossa \cite{Ferstl10}.


\subsection{Haasteita}
On pelätty, että uusien ja vähän koulutettujen opiskelijoiden käyttäminen ohjaajina ei olisi hyödyllistä opetettaville \cite{Harper02}. Ohjaajien puutteellinen kokemus tai koulutus opetuksesta voi vaikeuttaa ohjausta tai epäonnistumisen pelkoa ohjaajille \cite{Mark11}. Aiheesta kirjoitetut artikkelit kuitenkin väittävät, että hyvän ohjaajan ei tarvitse olla opinnoissaan kovin paljoa edellä opetettaviaan. Eri oppilaitosten välillä voi olla eroja siinä, kuinka päteviä tai luotettavia ohjaajat ovat tehtävissään \cite{Reges03}.
\\
Uusien opiskelijoiden palkkaaminen ohjaajiksi voi olla taloudellisesti epävarmaa, jos ohjaajat ilmoittavat työtuntinsa viikoittain \cite{Reges03}. Tämän vuoksi on vaikea etukäteen ennustaa tarkkaan, kuinka paljon ohjaajien käyttäminen tulee maksamaan, jolloin päättäjien vakuuttaminen ohjaajien käyttämisen eduista voi olla vaikeaa.
\\
Vaikka ohjaajat pääasiassa hyötyvät siitä, että he ovat suurin piirtein saman ikäisiä kuin opetettavansa, voi siitä seurata myös ongelmia tehtävien arvostelun suhteen \cite{Roberts95}. Koska ohjaajalla ei mahdollisesti ole paljon kokemusta tai auktoriteettia, voi hänen olla vaikea käsitellä opiskelijoita, jotka ovat mielestään saaneet liian huonon arvosanan. Lisäksi uusien ohjaajien voi olla vaikea arvioida, minkä arvosanan antaa epätavalliselle ratkaisulle. Ohjaajilla pitäisikin olla selvät ohjeet tehtävien pisteyttämisen suhteen omaperäisten ratkaisujen tai kiistatilanteiden varalta. Toisaalta jos pidetään tärkeänä sitä, että opiskelijat mieltävät ohjaajat vertaisekseen, voi olla parempi, että ohjaajat eivät osallistu tehtävien pisteyttämiseen ja opiskelijoiden arvosteluun \cite{Morgan02}. Jotkut opiskelijat voivat pitää epämiellyttävänä sitä, että heitä opettaa heitä itseään huomattavasti nuorempi ohjaaja \cite{Sperry08}.
\\
Ohjaajien palkitseminen opintopisteillä palkan sijaan on jakanut mielipiteitä. Vaikka muillakin tieteenaloilla opiskelijoita palkitaan opintopisteillä tutkimusprojekteihin osallistumisesta, on esitetty, että opettamiseen osallistuminen ei olisi yhtä arvokasta kuin tutkimukseen osallistuminen \cite{Reges88}.
\\
Jatkuva vaihtelu opettajan ja opiskelijan roolin välillä voi olla rasittavaa ohjaajalle \cite{Mark11}. Opettajakeskeiseen opetukseen tottuneen ohjaajan voi olla vaikeaa siirtyä opiskelijalähtöiseen oppimiseen.
\\
Lakisääteiset syyt voivat rajoittaa opiskelijoiden hyödyntämistä opetuksessa. Säädökset voivat esimerkiksi kieltää liian pienet kurssit. Vaikka jollakin tietojenkäsittelytieteen alkeiskurssilla olisi valtavat määrät opiskelijoita, jos heidät on jaettu pieniin ryhmiin, voi yksi iso kurssi tilastollisesti näyttää monelta pieneltä \cite{Reges03}. Säädökset voivat myös kieltää opiskelijoita arvostelemasta saman asteen opiskelijoiden tehtäviä varmistaakseen, että yliopistot eivät laiminlyö velvollisuuksiaan arvostelussa. Tätä sääntöä voi kiertää esimerkiksi arvostelemalla tehtävät laadullisesti siltä kannalta, kuinka ohjelma yleisesti ottaen toimii eikä keskittymällä koodin yksityiskohtiin \cite{Dickson11}. Tehtäviä voi tarkistaa myös automaattisesti tietokoneen avulla \cite{Vihavainen}.
\\
Koulutuksen ja kokemuksen puutteen takia ympäristön häiriöt voivat haitata ohjaajan työskentelyä koenutta opettajaa enemmän \cite{Patitsas12_3}. Esimerkiksi luokkahuoneen valaistus, pöytien järjestely ja opetustilanteen kesto vaikuttavat ohjaajan työhön.
\\
Aina ohjaajille ei ole tarjolla tarvittavaa koulutusta \cite{Shannon98}. Syitä koulutuksen puutteeseen voivat olla rahan tai henkilökunnan puute tai ajatus, että ohjaajat eivät tarvitse koulutusta. Henkilökunnalta voi puuttua kiinnostusta tai taitoa avustaa ohjaajia.



\section{Yhteenveto}
Useissa yliopistoissa on todettu, että opiskelijoiden käyttö opetuksessa nostaa opetuksen laatua ja mahdollistaa suurempien opiskelijamäärien opettamisen ilman, että kustannukset nousevat. Ohjaajana toimiminen antaa opiskelijalle hyödyllisiä taitoja ja kokemusta, ja ohjaajien läsnäolo kursseilla rentouttaa kurssin ilmapiiriä. Opiskelijoiden hyödyntämisen opetuksessa on havaittu toimivan niin suurissa kuin pienissäkin oppilaitoksissa.


\bibliographystyle{babplain}
\bibliography{lahteet}


\end{document}
