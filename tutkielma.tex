\documentclass[finnish]{tktltiki2}


\usepackage[utf8]{inputenc}
\usepackage[T1]{fontenc}
\usepackage{lmodern}
\usepackage{microtype}
\usepackage{amsfonts,amsmath,amssymb,amsthm,booktabs,color,enumitem,graphicx}
\usepackage[pdftex,hidelinks]{hyperref}

\makeatletter
\AtBeginDocument{\hypersetup{pdftitle = {\@title}, pdfauthor = {\@author}}}
\makeatother

\usepackage[fixlanguage]{babelbib}
\selectbiblanguage{finnish}

\usepackage[nottoc,numbib]{tocbibind}
\settocbibname{Lähteet}

\newtheorem{lau}{Lause}
\newtheorem{lem}[lau]{Lemma}
\newtheorem{kor}[lau]{Korollaari}

\theoremstyle{definition}
\newtheorem{maar}[lau]{Määritelmä}
\newtheorem{ong}{Ongelma}
\newtheorem{alg}[lau]{Algoritmi}
\newtheorem{esim}[lau]{Esimerkki}

\theoremstyle{remark}
\newtheorem*{huom}{Huomautus}

\title{Opiskelijoiden hyödyntäminen tietojenkäsittelytieteen \\ opetuksessa}
\author{Hanna Arpiainen}
\date{\today}
\level{Kandidaatin tutkielma}
\abstract{
Tietojenkäsittelytieteen alkeiskurssien opetuksen tarve on kasvanut ja opetuksen sisältöön on tullut muutoksia. Opetuksen laatu saattaa kärsiä resurssien puutteesta ja suuriin luentokursseihin liittyvistä ongelmista, kuten opiskelijoiden hukkumisesta massaan ja passiivisesta kuunteluun keskittyvästä opetustyylistä.

Käyttämällä opiskelijoita ohjaajina opetuksen laatua voi nostaa ilman, että opetuskustannukset kohoavat. Opiskelijoita voi käyttää erilaisiin työtehtäviin, kuten viikoittaisten keskustelusessioiden johtamiseen, tehtävien tarkistamiseen tai neuvonnan antamiseen. Taloudellisten ja opetuksen laatuun liittyvien hyötyjen lisäksi ohjaajilla on myönteinen vaikutus kurssin ja yliopiston ilmapiiriin, ja ohjaajat saavat toiminnastaan hyödyllistä kokemusta esiintymisestä, opettamisesta ja erilaisten ihmisten kanssa työskentelemisestä.
}

\keywords{Oppilaat opettajina, Tietojenkäsittelytieteen opetus}
\classification{Computer science education}


\begin{document}

\frontmatter
\maketitle
\makeabstract
\thispagestyle{empty}
\tableofcontents
\thispagestyle{empty}
\newpage

\mainmatter

\setcounter{page}{1}



\section{Johdanto}
Tietojenkäsittelytieteen opetuksen tarve kasvaa \cite{Roberts11}. Tarpeeseen vastaaminen on usein kuitenkin vaikeaa, sillä taloudellisten resurssien puute johtaa opiskelupaikkojen lukumäärän rajoittamiseen. Etenkin 1980-luvulla osaavista opettajista oli pulaa, jolloin henkilökuntaan jouduttiin palkkaamaan heikosti koulutettuja ulkopuolisia. Opettajapulaa aiheuttaa esimerkiksi teollisuuden menestys, joka houkuttelee opiskelijoita siirtymään nopeasti työelämään opintojen jatkamisen sijaan. Tällöin on yhä vaikeampi saada uusia opettajia uusille tietojenkäsittelytieteen opiskelijoille \cite{Roberts99}. \par

Resurssi- ja opettajapulasta seuraavaa ongelmaa voi helpottaa kehittämällä yhteistyötä teollisuuden kanssa, koska sillä on taloudellisia resursseja tukea opetusta \cite{Roberts11}. Resurssipulasta syntyvää ongelmaa voi ratkaista palkkaamalla mahdollisimman paljon opettamiseen keskittyvää henkilökuntaa tutkimukseen keskittyvän henkilökunnan sijaan. \par

Kolmas ratkaisu ongelmaan on opiskelijoiden hyödyntäminen opetuksessa. Monet yliopistot ovat päättäneet hyödyntää opiskelijoita kurssien opetukseen, sillä se muun muassa nostaa opetuksen laatua ja parantaa kurssin ilmapiiriä. Opiskelijoiden käyttö ohjaajina on myös edullisempaa kuin pitkälle koulutettu työvoima. \par

Tässä artikkelissa käsittelemme opiskelijoiden hyödyntämistä tie\-to\-jen\-kä\-sit\-te\-ly\-tie\-teen opetuksessa. Esittelemme, millä tavoin opiskelijoita voi käyttää alkeiskurssien ohjaajina, mitä työtehtäviä heillä tyypillisesti on ja millaista organisaatiota opiskelijoiden käyttäminen ohjaajina vaatii. Käsittelemme myös, mitä hyötyjä ja haasteita opiskelijoiden käyttämisestä on. \par


\section{Muutokset tietojenkäsittelytieteen opetuksessa}

Opiskelijamäärien kasvu, opetustyylien uudistuminen ja pula opettajista ovat aiheuttaneet muutoksia tie\-to\-jen\-kä\-sit\-te\-ly\-tie\-teen opetukseen. Lisääntyvistä opiskelijamääristä seuraava kurssikokojen kasvu voi tuoda ongelmia, jotka heikentävät opetuksen laatua.  \par

Tie\-to\-jen\-kä\-sit\-te\-ly\-tie\-teen opetuksen arvostus on kasvanut, eikä sitä enää pidetä toisarvoisena tutkimukselle \cite{Biggs07}. Monet yliopistot ovat todenneet laadukkaan opetuksen olevan keskeinen osa laitoksen toimintaa ja pyrkineet parantamaan opetusympäristöään. \par

Tietojenkäsittelytieteen opetuksen tyyli on muuttunut opetuskeskeisestä ajattelutavasta oppimiskeskeiseen \cite{Yadin11}. Op\-pi\-mis\-kes\-kei\-ses\-sä ajattelutavassa vastuu oppimisesta on opiskelijalla itsellään. Opiskelijolle ei enää vain kerrota tietoa, vaan he rakentavat itse osaamisensa vertaamalla uutta tietoa aiemmin opittuun. \par


\subsection{Opiskelijamäärien kasvu}
Tietojenkäsittelytieteen opiskelijoiden lukumäärän vaihtelu on syklistä \cite{Roberts11}. Esimerkiksi kotitietokoneiden ja World Wide Webin yleistyminen kasvattivat alan suosiota lisäten opiskelijoiden määrää, kun taas IT-kuplan puhkeaminen 2000-luvun alussa vähensi opiskelijamäärää. Viime vuosina opiskelijoiden määrä on ollut jälleen kasvussa. Esimerkiksi Yhdysvalloissa vuonna 2007 tie\-to\-jen\-kä\-sit\-te\-ly\-tie\-teen alkeiskursseilla oli keskimäärin noin 200 opiskelijaa laitosta kohti, kun taas vuonna 2011 heitä oli keskimäärin yli 300 \cite{TaulbeeReport}.  \par

Ajoittaisista taantumisista huolimatta opiskelijoiden määrä on pitkällä ajanjaksolla kasvanut, ja teollisuus tarvitsee yhä enemmän osaajia \cite{Roberts11}. Yksi syy osaajien tarpeeseen on se, että IT-kuplan puhkeamisen aiheuttamasta opiskelijakadosta seurasi pula tie\-to\-jen\-kä\-sit\-te\-ly\-tie\-teen koulutetuista ammattilaisista. Yhdysvaltojen yliopistojen pitäisi pystyä kaksin- tai kolminkertaistamaan tie\-to\-jen\-kä\-sit\-te\-ly\-tie\-teen valmistuneiden opiskelijoiden määrä, jotta alan uusiin työpaikkoihin ja eläköityvien ammattilaisten korvaamiseen riittäisi osaajia. \cite{Roberts11} \par

Tietojenkäsittelytieteen opetuksen tarvetta lisää myös se, että muiden aineiden opiskelijat haluavat opiskella tietojenkäsittelytieteen perusteita, koska he uskovat, että esimerkiksi ohjelmointitaidot auttavat heitä työmarkkinoilla \cite{Roberts11}. Etenkin alkeiskursseille pitäisi pystyä järjestämään riittävästi opetustarjontaa. \par

Kasvavista opiskelijamääristä voi seurata ongelmia. Kurssikoon kasvaessa kasvaa myös opiskelijoiden henkilökohtaiseen ohjaukseen tarvittava työmäärä \cite{Kay98}. Suurissa opiskelijamäärissä parhaiden ja heikoimpien opiskelijoiden ero voi olla hyvin suuri, joten ääripäät tarvitsisivat erityisen paljon yksilöllistä huomiota. Monilla uusilla opiskelijoilla ei ole tietojenkäsittelytieteestä aiempaa kokemusta, jolloin suuret luennot ja yksin tehtävät harjoitukset voivat tuottaa heille hankaluuksia \cite{Murphy11}. \par

Suurten ryhmäkokojen seurauksena opiskelijat uhkaavat hukkua massaan ja voivat sen seurauksena menettää mielenkiintonsa alaa kohtaan \cite{Kay98}. Se, että opiskelija pääsee opiskeluun mukaan heti opintojen alussa, ehkäisee syrjäytymistä \cite{Settle12}. Pelkkä luentojen passiivinen kuuntelu ei kuitenkaan rohkaise opiskelijoita osallistumaan \cite{Kopp00}. Aktiivinen osallistuminen ehkäisisi opintojen kesken jättämistä \cite{Boyer07}, mutta yliopisto-opintojen itsenäisyys on uutta monille opiskelijoille, ja heidän on helppo oppia huonoja opiskelutapoja ja olla välittämättä läsnäolon tärkeydestä kursseilla \cite{Kopp00}. Luentoihin painottuvilla kursseilla voi olla vaikeuksia tarjota opiskelijoille roolimalleja, jotka näyttäisivät esimerkillään, kuinka opinnoissa pysyy mukana. Uusien opiskelijoiden voi olla hankala lähestyä kokenutta luennoijaa \cite{Kopp00}. \par

Ryhmäkoolla voi olla vaikutusta opiskelijoiden itseluottamukseen \cite{Boyer07}. Suurten luentokurssien opiskelijoilla voi olla heikompi itseluottamus omien kykyjensä tai opintomenestyksensä suhteen kuin pienryhmissä opiskelevilla, ja he voivat olla epävarmempia sen suhteen, kuinka hyödyllistä tietojenkäsittelytieteen osaaminen on heille tulevaisuudessa. \par

Suurten luentojen opetustyyli voi vaikeuttaa oppimista. Suurissa ryhmäkoissa opiskelijoilla voi esimerkiksi olla vaikeuksia keskittyä ja seurata opetusta \cite{Boyer07}. Luentojen passiivisella kuuntelemisella ja asioiden ulkoa opettelulla opiskelija voi yltää pintapuoliseen oppimiseen, mutta syvempään oppimiseen vaaditaan aktiivista opiskelua \cite{Boyer07}. Pintapuolisella oppimisella tarkoitetaan opiskelutapaa, jossa opiskelija keskittyy yksityiskohtiin kokonaisuuden sijasta suoriutuakseen yksittäisistä tehtävistä \cite{DeepSurfaceLearning}. Opiskelu voi olla ylimalkaista eikä välttämättä johda opeteltavan aiheen ymmärtämiseen. Syvässä oppimisessa opiskelija yhdistelee itse aktiivisesti aiemmin oppimaansa ja pyrkii ymmärtämään käsiteltävän asian. Syvässä oppimisessa opiskelija pyrkii kokonaisvaltaiseen ymmärtämiseen, mikä auttaa kehittämään kriittistä ajattelua ja käsitteiden pitkäaikaista muistamista. \par

Tehtävien tekemiseen tarvittava työmäärä kasvaa kurssikoon kasvaessa \cite{Kay98}. Jos kokeista ja tehtävistä halutaan saada mahdollisimman nopeasti tarkistettavia, niiden tekeminen vaikeutuu. Tehtävät tulisi olla määritelty mahdollisimman selkeästi, sillä epäselvien kysymysten aiheuttamia väärinymmärryksiä tulee suurten opiskelijamäärien kanssa enemmän. \cite{Kay98} \par

Kurssikoon kasvaessa kurssin hallinnointi vaikeutuu, mukaan lukien kurssinateriaalin valmistelu ja jakaminen opiskelijoille, neuvonnan tarjoaminen ja henkilökunnan hallinnointi \cite{Chamillard02}. Suurella kurssilla voi olla haasteellista taata, että kaikki opiskelijat arvostellaan tasavertaisesti. \par



\subsection{Tietojenkäsittelytieteen alan kehittyminen}

Ketterän ohjelmistokehityksen yleistymisen takia tarvitaan yhä enenevässä määrin ohjelmistoalan ammattilaisia, jotka osaavat toimia ohjaajina ohjelmoijille ja ohjelmointitiimeille \cite{Vikberg}. Tämän takia oppilaitosten on annettava tietojenkäsittelytieteen opiskelijoille mahdollisuus opetella ohjelmoinnin lisäksi ohjaajana toimimiseen liittyviä taitoja. Perinteinen passiivinen luentojen kuuntelu ei vastaa tietojenkäsittelytieteen alan työelämää, jossa tarvitaan ongelmanratkaisutaitoja ja kykyä toimia ryhmässä \cite{Vihavainen}. \par

Teknologian kehitys on muuttanut oppimisympäristöä, ja opetusta voi tarjota virtuaalimaailmassa \cite{Yadin11}. Tehtävien tarkastusta voidaan nopeuttaa automatisoimalla se, ja luentoja ja yksilöllistä ohjausta voidaan välittää sähköisesti \cite{Jimenez-Periz00}. \par



\subsection{Pula opettajista}
Tietojenkäsittelytieteen alaa vaivaa puute pätevistä opettajista, mikä vaikeuttaa kasvaviin opiskelijamääriin vastaamista \cite{Roberts99}. Opettajapulan syy on menestyvä teollisuus, joka houkuttelee opiskelijat jättämään opintonsa ja siirtymään töihin. Kun opiskelijat siirtyvät työelämään akateemisen alan valitsemisen sijaan, uusien opettajien löytäminen on yhä vaikeampaa. \par

Uusien opettajien saamisen lisäksi nykyisten opettajien säilyttäminen voi olla vaikeaa. Työ\-voi\-ma\-pu\-la aiheuttaa ylitöitä ja stressiä opettajille, mikä lisää heidän halukkuuttaan siirtyä muihin töihin \cite{Roberts99}. Yliopiston päätoimiset opettajat voivat olla kiinnostuneempia opettamaan syventäviä kursseja tai osallistumaan tutkimukseen alkeiskurssien opettamisen sijaan \cite{Shannon98}. \par




\section{Oppilaat opettajina}
Kasvaviin opiskelijamääriin, suuriin luentokursseihin ja työvoimapulaan liittyviä ongelmia voi ratkaista käyttämällä opiskelijoita tietojenkäsittelytieteen opetuksessa. Opiskelijat voivat toimia ohjaajana ryhmälle alkeiskurssin o\-pis\-ke\-li\-joi\-ta \cite{Reges88}. Opiskelijoiden ei tarvitse olla vastuussa jonkin tietyn ryhmän ohjaamisesta, vaan he voivat toimia yleisinä ohjaajina ja antaa neuvontaa sitä tarvitseville opiskelijoille \cite{Vikberg, Vihavainen}. Ohjaajien tehtävä on tarjota lisäopetusta kurssin tueksi ja auttaa kurssin opiskelijoita esimerkiksi tehtävien kanssa \cite{Patitsas12_3}. \par

Ryhmänohjaajina työskentelyn lisäksi opiskelijat voivat toimia vertaistukena \cite{Tashakkori05}. Vertaistukea voi saada opiskelija, joka tarvitsee lisää tukea opintoihinsa, tai joka tarvitsee ylimääräisiä haasteita. Tukihenkilön tulee osata opettamansa kurssin sisältö hyvin. Tukihenkilö tapaa neuvottavaansa ja avustaa häntä tehtävien kanssa. Neuvonta voi mennä myös ristiin; opiskelija voi olla tukihenkilönä toiselle opiskelijalle yhdellä kurssilla ja saman henkilön neuvottavana jollakin toisella. \par

Opiskelijoiden käyttäminen ohjaajina vaatii laitokselta sitoutumista \cite{Kopp00}. Koska ohjaajien käyttämiseen tarvitaan ennakkoinvestointeja, kannattaa laitoksen järjestää ohjaajia vain kursseille, joita siellä on aiemmin opetettu. \par

Esimerkiksi Helsingin yliopistossa jopa 20 prosenttia tie\-to\-jen\-kä\-sit\-te\-ly\-tie\-teen opiskelijoista toimii ohjaajana alkeiskursseilla \cite{Vihavainen}. Joissakin laitoksissa, esimerkiksi Kalifornian yliopistossa, ohjaajana toimiminen voi olla pakollinen osa tutkintoa \cite{Kay95}.  \par


\subsection{Ohjaajien työtehtäviä}
Ohjaajilla voi olla keskeinen osa tietojenkäsittelytieteen opetuksessa ja heitä voi käyttää moniin erilaisiin tehtäviin. Joissakin laitoksissa opiskelijat käyttävät enemmän aikaa ohjaajien kuin luennoijan kanssa \cite{Patitsas12_3}.\par

Ohjaajien tehtäviin kuuluu esimerkiksi viikottaisten keskustelusessioiden pitäminen opiskelijoille \cite{Reges88}. Keskustelusessioissa käydään läpi kurssin sisältöä, mutta niiden ei tarvitse noudattaa mitään tiukkaa kaavaa, vaan ohjaaja voi muokata ne vastaamaan opiskelijoiden tarpeita. Ohjaaja voi esimerkiksi kerrata luentojen asioita, esittää lisäesimerkkejä, tai vastata opiskelijoiden esittämiin kysymyksiin. Viikkotapaamisten ei tarvitse olla vain ohjaajan esiintymistä, vaan ne voivat sisältää opiskelijoiden ryhmäkeskusteluja \cite{Decker06}. Tarvittaessa ohjaajan on oltava valmis hylkäämään aiemmat suunnitelmansa, mikäli selviää, että opiskelijat ovat ymmärtäneet jonkin tärkeän asian väärin ja tilanne on korjattava \cite{Reges88}. \par

Jos ohjaajalla on vastuullaan tietty ryhmä opiskelijoita, hänen tehtäviinsä voi kuulua viikottainen yksilöllinen palautekeskustelu jokaisen ryhmänsä jäsenen kanssa \cite{Reges88,Reges03}. Keskustelutilaisuudessa opiskelija voi selittää ratkaisunsa johonkin tehtävään, tai kertoa, että ei ole ymmärtänyt jotakin kurssin asiaa. Ohjaaja taas voi auttaa opiskelijaa ymmärtämään, mistä jokin virhe johtuu, ja miten sen voi jatkossa välttää. Näissä keskusteluissa ohjaaja saa kuvan opiskelijan edistymisestä kurssilla.  \par

Ohjelmointikursseilla ohjaajat voivat päivystää tietokoneluokassa valmiina auttamaan kurssin opiskelijoita kurssiin liittyvissä ongelmissa. Ohjaajan ei tulisi antaa suoraa vastausta tehtävään, vaan ohjata ja neuvoa opetettavaa löytämään ratkaisu itse \cite{Vikberg, Kurhila11}. Ohjelmointitehtävien yhteydessä ohjaaja voi kannustaa opetettavaa kirjoittamaan hyvää ja luettavaa koodia esimerkiksi neuvomalla tätä nimeämään muuttujat selkeästi. Ruuhka-ajoille päivystäviä ohjaajia pyritään järjestämään enemmän \cite{Reges88, Reges03}, ja ohjaajat saattavat voida kutsua lisäapua, mikäli tietokoneluokka ruuhkautuu pahasti \cite{Kurhila11}. Ohjaajat voivat erottuakseen joukosta täydessä tietokoneluokassa käyttää esimerkiksi turvaliivejä \cite{Vihavainen}. Yleisen neuvonnan lisäksi ohjaajalla voi olla yksityinen vastaanottoaika, jonka aikana opiskelijat voivat käydä kysymässä häneltä yksilöllisiä vastauksia kurssiin liittyviin tehtäviin \cite{Decker06}. \par

Ohjaajat voivat pitää ennen koetta kertaustilaisuuden kurssin sisällöstä. Kertaustilaisuudessa he voivat kertoa, mitkä kurssin asiat ovat tärkeitä ja mitä kokeessa todennäköisesti kysytään, tai esitellä vanhoja kokeita ja niiden ratkaisuja \cite{Decker06}. Ohjaajat voivat myös valvoa kurssikokeita \cite{Richards00}. \par

Ohjaajan tehtäviin kuuluu usein tehtävien tarkistaminen \cite{Dickson11}. Kun ohjaajat osallistuvat tehtävien tarkastamiseen, voi tehtäviä olla enemmän, ja luennoija voi käyttää aikansa tehokkaammin esimerkiksi kirjoittamalla kommentteja tehtäviin. \par

Jotta ohjaaja pysyisi mahdollisimman hyvin perillä kurssin etenemisestä ja käsitellyistä esimerkeistä, hänen voidaan edellyttää osallistuvan kurssin luennoille \cite{Reges03, Decker06}. Jos ohjaajat käyvät kurssin materiaalin läpi jo ennen kuin luennoija esittää sen opiskelijoille, he voivat huomauttaa mahdollisista virheistä tai puutteista \cite{Vikberg}. Ohjaajat saattavat ratkaista kurssin tehtävät ennen kuin ne annetaan opiskelijoille, mikä auttaa heitä valmistautumaan opiskelijoiden neuvontaan \cite{Vihavainen}. \par

Ohjaaja voi päivittää kurssin verkkosivuja \cite{Dickson11}. Tämä vähentää luennoijan työtaakkaa ja auttaa opiskelijoita, joilla on vaikeuksia omien muistiinpanojen tekemisen kanssa. Luennoille osallistuva ohjaaja voi päivittää esimerkkejä verkkosivuille reaaliajassa. \par





\subsection{Ohjaajien koulutus}

Uusien ohjaajien kouluttamiseen voi kuulua erillinen tietojenkäsittelytieteen opettamiseen keskittyvä kurssi \cite{Reges88, Roberts95} tai aloitusseminaari \cite{Sperry08}. Myös teollisuuden edustajat voivat olla kiinnostuneita kouluttamaan ohjaajia \cite{Morgan02}. \par

Ohjaajilla voi olla paljon koulutusta opettamansa kurssin alussa, ja koulutustapaamiset vähenevät ja lopulta loppuvat kokonaan kurssin edetessä \cite{Roberts95}. Aina muodollista koulutusta ei juuri tarvita tai se ei ole mahdollista, jos uusia ohjaajia hankitaan jatkuvasti opetuksen edetessä \cite{Kurhila11}. Tällöin ohjaajat oppivat työtä tekemällä ja luennoijan ja opiskelijoiden palautteen avulla \cite{Shannon98, Vihavainen}. Lisäksi kokeneet ohjaajat voivat kouluttaa epävirallisesti uusia ohjaajia neuvomalla heitä \cite{Kurhila11}. Vaikka ohjaamisesta ei olisi varsinaista koulutusta, ohjaajana toimimisen palkitseminen opintopisteillä antaa opiskelijoille kuvan siitä, että kokemuksesta on tarkoitus oppia \cite{Vikberg}. \par

Ohjaajien kouluttamiseen liittyy usein keskustelutilaisuuksia. Niissä voidaan esimerkiksi harjoitella tehtävien pisteyttämistä, vaikeiden käsitteiden opettamista tai hankalien opiskelijoiden käsittelyä \cite{Reges03}. Kes\-kus\-te\-lu\-ti\-lai\-suuk\-sis\-sa voidaan käydä läpi erilaisia oppimistyylejä ja keskustella, millainen on hyvä tai huono ohjaaja \cite{Kay95}. Uudet ohjaajat voivat pitää harjoitusesitelmiä toisilleen, mistä he saavat esiintymiskokemusta ja palautetta. Useita erilaisia esitelmiä nähdessään ohjaajat voivat oppia uusia lähestymistapoja aiheeseen.  \par

Ohjaajien koulutukseen voi kuulua tehtävien ja kokeiden tekemisen harjoittelu \cite{Kay95}. Ohjaajille voidaan antaa tehtäväksi keskustella, millaisen tie\-to\-jen\-kä\-sit\-te\-ly\-tie\-teen alkeiskurssin he suunnittelisivat. Kurssin suunnitteleminen voi saada ohjaajat ajattelemaan tie\-to\-jen\-kä\-sit\-te\-ly\-tie\-teen opetukseen liittyviä laajoja kysymyksiä, kuten mitä sivuaineopiskelijoiden tulisi alasta oppia, tai millaiset pohjatiedot uusilla opiskelijoilla voi olettaa tietojenkäsittelytieteestä olevan.  \par

Kokeneet ohjaajat voivat osallistua uusien ohjaajien koulutukseen kerratakseen omia taitojaan, ja samalla he voivat jakaa kokemustaan ja toimia roolimalleina uusille ohjaajille \cite{Reges88}. Kokeneiden ohjaajien tarjoama vertaisopetus ja rohkaisu nopeuttaa uusien ohjaajien koulutusta \cite{Decker06}. Opiskellessaan opetusta ryhmänä ohjaajat oppivat tuntemaan toisensa ja voivat ystävystyä \cite{Roberts95}. \par

Uusien ohjaajien tukemisen lisäksi vanhat ohjaajat voivat antaa hyödyllistä palautetta ohjaajien koulutuksesta \cite{Decker06}. Ohjaajat voivat suositella opettamiaan opiskelijoita uusiksi ohjaajiksi ja ehdottaa, millaisilla valintaperusteilla saadaan valittua parhaat ohjaajaehdokkaat.  \par

Kehittyäkseen työssään ohjaajien on tärkeää saada palautetta \cite{Patitsas12, Patitsas12_3}. Myönteinen palaute parantaa ohjaajan itseluottamusta, mikä puolestaan parantaa hänen opetustaitojaan. \par





\subsection{Käytännön organisaatio}

Ohjaajia hyödyntävien kurssien henkilökunnan rakenne vaihtelee kursseittain ja yliopistoittain. Esimerkiksi pienessä yliopistossa voi toimia malli, jossa luennoijan lisäksi kurssin henkilökuntaan kuuluu vain muutama ohjaaja \cite{Dickson11}, kun taas suurissa laitoksissa kursseilla, joilla on paljon opiskelijoita, on hyödyllistä, että kurssin henkilökuntaan kuuluu ohjaajien työtä hallinnoiva assistentti \cite{Reges03}. Assistentti toimii ohjaajien ja luennoijan välisenä kontaktina ja jakaa kurssin materiaalin eteenpäin ohjaajille. Hän voi auttaa luennoijaa kurssin sisällön kanssa. Assistentti voi olla joku kurssin ohjaajista tai erillinen tehtävään palkattu henkilö. \par

Mikäli henkilökuntaa tarvitaan kursseille paljon, ohjaajien käyttäminen tarvitsee usein toimiakseen jonkinlaisen koordinaattorin \cite{Roberts95}. Siinä missä assistentin tehtävä on välittää kurssiin liittyvää tietoa tietyn kurssin luennoijan ja ohjaajien välillä, koordinaattorin kuuluu pitää huolta siitä, että ohjaajatoiminnan rakenne toimii, eli esimerkiksi varmistaa ohjaajien palkkaus ja koulutus. Koordinaattori voi olla esimerkiksi opinnoissaan pitkälle edennyt opiskelija, joka on aiemmin toiminut ohjaajana. Jos ohjaajia on paljon, voidaan tarvita useampia koordinaattoreita. Ohjaajatoiminnan laadun takaamiseksi jonkun tiedekunnan jäsenen tulisi olla ylimmässä vastuussa koordinaattorien työstä. \par

Koordinaattorin tehtäviin kuuluu kurssin käytännön hallinnointi, eli esimerkiksi luokkahuoneiden varaaminen ja tarvittavien ohjaajien jakaminen kursseille. He myös hoitavat ohjaajien valinnan, palkkaamisen ja kouluttamisen \cite{Reges88,Roberts95}. Koordinaattori voi huomauttaa ohjaajan puutteellisesta toiminnasta \cite{Reges88}. \par

Koordinaattorien kuuluu ylläpitää tiedon kulkua ja kommunikointia henkilökunnan välillä. Heidän tulee myös järjestää tarvittavat kommunikointikanavat opiskelijoiden ja ohjaajien välille \cite{Reges88}. Koordinaattorien tehtäviin kuuluu järjestää tapaamisia henkilökunnalle, jotta luennoija saa kuvan opiskelijoiden etenemisestä. Viikkopalaverissa ohjaajat voivat kertoa mahdollisista ongelmista, ja luennoija voi tarpeen tullen hidastaa opetustahtia tai selittää jonkin epäselväksi jääneen asian uudestaan. Palaverissa ohjaajat pääsevät tapaamaan toisiaan ja jakamaan tietoa. \par

Koordinaattorien tehtäviin kuuluu usein myös järjestää tapaamisia, joissa käydään läpi opetukseen liittyviä asioita \cite{Reges88, Roberts95}. Koordinaattori voi kerätä opiskelijoilta palautetta ohjaustilaisuuksista, ja esitellä palautteen ohjaajille henkilökunnan tapaamisessa \cite{Patitsas12_2}. Palautteen avulla koordinaattori voi suunnitella parannuksia seuraavalle lukukaudelle. Ohjaajan voidaan edellyttää kirjoittavan kurssin koordinaattorille työstään viikottainen raportti, jossa hän voi kertoa, mikä viikon opetuksessa meni hyvin ja mikä huonosti, ja miten kurssia voidaan parantaa \cite{Hug11}. \par


\subsection{Hyvän ohjaajan ominaisuuksia}
Ohjaajalta vaaditaan innostusta auttaa muita oppimaan ja vastuullisuutta noudattaa kurssin käytäntöjä \cite{Reges88}. Ohjaajan ei ole välttämätöntä olla esimerkiksi ohjelmoinnin erityisosaaja, mutta hänen edellytetään ymmärtävän kurssin sisältö ja olevan valmis kehittämään itseään pystyäkseen auttamaan opetettaviaan riittävästi. Ohjaaja voi hyötyä siitä, että hänellä on itsellään ollut oppimisvaikeuksia, sillä se voi auttaa häntä ymmärtämään paremmin opetettaviensa vaikeuksia samojen tehtävien kanssa \cite{Decker06}.  Aluksi heikolta ohjaajalta vaikuttava ehdokas voi osoittautua hyväksi ohjaajaksi, jos hänellä on innostusta kehittyä työssään tai hän saa enemmän itseluottamusta kurssin edetessä \cite{Dickson11}. \par

Aito kiinnostus opetettavien auttamiseen on keskeistä ohjaajan menestykselle \cite{Richards00}. Ohjaaja, joka on töissä vain palkan tai velvollisuuden vuoksi ja pyrkii pelkkään minimisuoritukseen, ei tue oppimista \cite{Richards00}. Huono ohjaaja ei valmistaudu opetustilaisuuteen riittävästi tai vähättelee opetettaviaan tai opetettavaa materiaalia. Opiskelijat tuntevat tulevansa kohdelluksi epäoikeudenmukaisesti, jos he tietävät, että muilla ryhmillä on parempi ohjaaja kuin heillä. \par

Ohjaajan voi edellyttää olevan aktiivinen, vaikka opetettavat eivät kysyisi mitään \cite{Vikberg}. Ohjaajan tulee osata huomioida erilaisista taustoista tulevat opiskelijat \cite{Kay98}. Jos ohjaajalla on vastuullaan jokin tietty ryhmä opiskelijoita, hänen tulisi opetella heidän nimensä ja tutustuttava heihin \cite{Bernstein}. \par

Pareittain työskenteleminen parantaa ohjaajien suoritusta \cite{Patitsas12_3}. Ohjaaja sitoutuu tehtäväänsä paremmin, jos hänellä on pari, joka odottaa, että työtehtävät jaetaan tasavertaisesti. Työparin läsnäolo luo turvallisuuden tunnetta, jos ohjaaja voi luottaa siihen, että pari auttaa ja tukee tarvittaessa. Pareittain työskentely parantaa ohjaajien viihtymistä työssään. \par

Ohjaajan olisi osattava opetuskieli hyvin \cite{Richards00}. Kielivaikeuksista seuraa kom\-mu\-ni\-kaa\-tio-on\-gel\-mia, minkä takia opiskelijat eivät välttämättä ymmärrä ohjaajaa. \par

Uudet opiskelijat ovat yleensä parempia ohjaajia kuin vanhemmat, opinnoissaan pidemmälle edenneet opiskelijat \cite{Dickson11}. Uusilla ohjaajilla on muistissaan hyvä kuva kurssin pääasioista, koska he ovat itse käyneet kurssin vasta vähän aikaa sitten. Uusi ohjaaja oppii nopeasti, millainen on hyvä ohjaaja, mikäli hän on juuri itse ollut ohjattavana \cite{Vihavainen, Vikberg}. Ohjattavien voi olla vaikea suhtautua ohjaajaan, jos tämä on heitä kovin paljon vanhempi \cite{Decker06}. \par

Pitkälle edenneiden opiskelijoiden saattaa olla vaikeaa ymmärtää kurssin materiaalia, opetustekniikkaa tai alkeiskurssien opiskelijoiden ongelmia, ja heidän voi olla vaikeampi ymmärtää laitoksen tietokoneita \cite{Reges88}. Korkealla tutkintoarvolla ei välttämättä ole paljoa yhteistä opetuksen kanssa, ja materiaalin osaaminen on vain yksi osa opetuksesta, joten pitkälle edenneet opinnot eivät takaa menestystä opettajana \cite{Baldwin00}. Opinnoissaan pitkälle edenneitä opiskelijoita voi myös olla vaikea houkutella ohjaajiksi, koska tietojenkäsittelytieteen alalla on tarjolla paljon työmahdollisuuksia \cite{Kay98}.  \par

Toisaalta tohtorin tai maisterin tutkinnon suorittanutta henkilökuntaa voi pitää uusia ja kouluttamattomia opiskelijoita parempana vaihtoehtona ohjaajaksi tai opettajaksi \cite{Baldwin00}. Kokenut opettaja voi helpommin viitata eri kurssien sisältöihin ja kertoa, miten kurssit liittyvät toisiinsa. Näin hän voi auttaa opiskelijoita näkemään materiaalin laajassa mittakaavassa eikä vain kokelmana erillisiä kursseja. Laajat pohjatiedot voivat auttaa opettajaa tunnistamaan ja omaksumaan nopeasti pääkohdat materiaalista, jota hän ei tunne ennalta. \par

Pitkälle koulutettu henkilökunta voi myös toimia roolimallina opiskelijoille, sillä tutkinnon suorittaminen viittaa kykyyn sitoutua tutkimukseen sekä ajatella luovasti ja itsenäisesti \cite{Baldwin00}. Kokenut opettaja kannustaa opiskelijoita jatkamaan opinnoissaan ja ajattelemaan tietojenkäsittelytiedettä kiinnostavana tieteenalana teollisuuteen siirtymisen sijaan, mikä auttaa opettajapulaan. Akateemisen uran valitsevat opiskelijat tarvitsevat opinnoista asiantuntevaa neuvontaa, jota osaa parhaiten antaa ohjaaja, jolla on kokemusta aiheesta. \par






\section{Ohjaajien käytön seurauksia}

Opiskelijoiden hyödyntäminen tietojenkäsittelytieteen opetuksessa on mahdollistanut suurempien opiskelijamäärien opettamisen ilman, että kustannukset kohoavat \cite{Reges88}. Samalla opetuksen laatu on noussut, koska opiskelijat saavat aiempaa enemmän yksilöllistä huomiota. Ohjaajilla on myönteinen vaikutus kurssin ja koko yliopiston ilmapiiriin \cite{Dickson11,Roberts95, Tashakkori05}. \par

Vaikka opiskelijoiden käytöstä ohjaajina on havaittu olevan paljon hyötyä, siihen liittyy myös haasteita ja ennakkoluuloja.  \par

\subsection{Hyötyjä}

Opiskelijoiden käyttämisestä ohjaajina on taloudellista etua yliopistolle, sillä heille maksettava palkka on yleensä matalampi kuin kokeneemmille assistenteille, tai heidät palkitaan opintopisteillä \cite{Reges88}. Esimerkiksi Arizonan yliopistossa uusien opiskelijoiden palkkaaminen ohjaajiksi kokeneiden assistenttien sijaan laski kustannuksia 85 000 dollarista 75 000 dollariin. Hankkeen alkuvaiheessa säästöt olivat jopa 30 prosenttia, koska päättäjät oli saatava vakuutettua siitä, että ohjaajien palkkaamiseen olisi varaa. Helsingin yliopistossa on havaittu, että ohjaajien hyödyntäminen skaalautuu hyvin myös suurille opiskelijamäärille \cite{Kurhila11}. Palkkaamalla enemmän ohjaajia luentoja voitiin karsia tai jättää kokonaan pois. \par

Uusien opiskelijoiden käyttäminen ohjaajina on parantanut opiskelijoiden arvosanoja tietojenkäsittelytieteen alkeiskursseilla \cite{Decker06, Kurhila11}. Kasvattamalla ohjaajien määrää henkilökohtaisen ohjauksen määrää voi lisätä ja luentoja vähentää tai poistaa kokonaan \cite{Kurhila11}. \par

Koska opiskelijoiden palkkaaminen ohjaajiksi on halvempaa kuin muiden vaihtoehtojen, voi heitä palkata enemmän. Näin opetettavat saavat enemmän yksilöllistä huomiota. Kurssilla voi olla enemmän pieniä tehtäviä, jotka innostavat opiskelijoita paremmin lukemaan materiaalia, kun tehtävien tarkastamiseen on käytössä enemmän henkilökuntaa \cite{Dickson11}. Kun ohjaajia on luokassa useita, opiskelijat saavat huomiota nopeammin \cite{Ferstl10}. Tehtävien suuri määrä vaikuttaa myös oppimiseen, sillä opiskelija oppii materiaalin sitä paremmin, mitä enemmän hän pääsee harjoittelemaan sitä käytännön tehtävissä \cite{Vikberg}. Jonkun tietyn henkilökunnan edustajan mahdollinen puolueellisuus ei välttämättä pääse vaikuttamaan opiskelijoihin, jos luokassa on samaan aikaan useita erilaisia ohjaajia \cite{Morgan02}.  \par

Ohjaajien käyttäminen voi kehittää kurssia, sillä ohjaajat pystyvät antamaan hyödyllistä palautetta kurssin toiminnasta ja ehdottamaan parannuksia \cite{Decker06} tai huomauttamaan, jos materiaalissa on puutteita \cite{Dickson11}. Ohjaaja voi olla jonkin tietojenkäsittelyn osa-alueen erityisosaaja ja tuoda osaamisensa kurssin hyödyksi \cite{Dickson11}. Aiheesta aidosti kiinnostuneilla opiskelijoilla voi olla luennoijaa enemmän aikaa omistautua kurssille \cite{Dickson11}. Kun luennoija esittelee kurssin sisällön ohjaajille, hän saa tilaisuuden käydä opetettava materiaali läpi ennen sen esittämistä luennolla \cite{Kopp00}. Ohjaajia hyödyntämällä kurssia voi kehittää ketterästi, kun ohjaajat voivat päivittää kurssimateriaalia silloin, kun luennoijalla ei ole aikaa siihen tai jos hän ei ole huomannut sen olevan jäljessä \cite{Dickson11}. \par

Yliopistolla työskentelevät ohjaajat voivat toimia roolimalleina muille opiskelijoille \cite{Roberts95, Tashakkori05}, sillä opiskelijat tietävät, että ohjaajaksi pääsemiseen vaaditaan kelvollista menestymistä opinnoissa \cite{Bernstein}. Opiskelijoiden voi olla vaikea nähdä itsensä esimerkiksi kokeneena luennoijana, kun taas ohjaaja tarjoaa helpommin lähestyttävämmän roolimallin \cite{Roberts02}. Jos ohjaajat valitaan opiskelijoiden keskuudesta, voidaan henkilökunnasta saada aiempaa monipuolisempi ja siten tarjota paljon erilaisia roolimalleja \cite{Morgan02}. Etenkin naisille ja vähemmistöjen edustajille voi olla tärkeää saada kaltaisensa roolimalli, jotta heidän on helpompi tuntea kuuluvansa joukkoon. Vähemmistöopiskelijat voivat jäädä opiskelun ulkopuolisten epämuodollisten ryhmien ulkopuolelle. Ohjaajien käyttäminen mahdollistaa paremmin pienryhmissä opiskelun, joka auttaa opiskelijoita tutustumaan toisiinsa. Yhteenkuuluvuuden tunne saa opiskelijat pysymään alalla paremmin. \par

Kurssin ilmapiiri rentoutuu, jos opiskelijat näkevät luennoijan ja ohjaajan tulevan hyvin toimeen keskenään \cite{Dickson11}. Mikäli ohjaaja on läsnä esimerkiksi luennolla ja esittää luennoijalle kysymyksiä aiheesta, rohkaisee se opiskelijoitakin osallistumaan enemmän luennoilla. Kurssin opiskelijat hyötyvät, jos heidän keskuudessaan on ohjaaja, joka osaa heti kurssin alussa kertoa luennoijan opetustyylistä ja painotuksista. Motivoitunut ohjaaja motivoi opetettaviakin työskentelemään ahkerammin. Kurssin opiskelijoiden voi olla helpompi antaa palautetta ohjaajalle kuin luennoijalle \cite{Morgan02}. \par

Ohjaajana toimiminen voi innostaa opiskelijoita pyrkimään luennoijiksi \cite{Morgan02}. Ohjaamisesta saatavan opettamiskokemuksen ansiosta ohjaajana toimiminen voi olla koulutusta myöhempiä opettajan tehtäviä varten \cite{Roberts95}. Ohjaajia ohjaavat opettajat saavat tilaisuuden vaikuttaa ohjaajien näkemykseen opetusalasta \cite{Morgan02}. \par

Ohjaajana toimisesta on hyötyä myös ohjaajille itselleen. Ohjaajat saavat palkkaa tai opintopisteitä, minkä lisäksi ohjaajana toimiminen on hyvin opettavainen kokemus \cite{Reges03}. Ohjaajana toimiminen parantaa ohjaajan ryhmätyö- ja esiintymistaitoja \cite{Reges03} ja opettaa vastuunkantoa \cite{Dickson11} ja ajanhallintaa \cite{Aminzadeh}. Ohjaajana toimiessaan oppii keskustelemaan tie\-to\-ko\-ne\-tai\-doil\-taan eritasoisten opiskelijoiden kanssa \cite{Vikberg}. Koska ohjaajat yleensä ovat tekemisissä luennoijan kanssa tavallisia opiskelijoita enemmän, heillä on tilaisuus saada ylimääräistä ohjausta opintoihinsa \cite{Aminzadeh}. \par

Opettaessaan ohjaajat oppivat itsekin materiaalin paremmin \cite{Reges03}, koska jonkin asian osaa kunnolla vasta sitten, kun osaa opettaa sen toiselle \cite{Bernstein, Dickson11}. Opetettavien kannustaminen hyvään ohjelmointityyliin parantaa ohjaajankin ohjelmointityyliä \cite{Roberts95}. Opettamalla muita ohjaaja voi oppia omista opiskelutottumuksistaan \cite{Ferstl10}.  \par

Opettaessaan muita ohjaaja oppii paremmin ymmärtämään tarpeen jatkuvalle elinikäiselle opiskelulle \cite{Paxton05}. Ohjaajana toimimisesta voi saada hyvän mielen päästessään auttamaan muita \cite{Aminzadeh} ja saadessaan nähdä opetettaviensa edistyvän \cite{Kurhila11}. Ohjaajat voivat saada toisistaan seuraa ja osallistua keskenään erilaisiin vapaa-ajan aktiviteetteihin \cite{Roberts95}, ja ohjaajana toimiminen liittää opiskelijan paremmin mukaan yliopiston yhteisöön \cite{Dickson11}. Kokemus ohjaajana toimimisesta näyttää myös hyvältä ansioluettelossa \cite{Ferstl10}. \par




\subsection{Haasteita ja ennakkoluuloja}
On pelätty, että uusien ja vähän koulutettujen opiskelijoiden käyttäminen ohjaajina ei olisi hyödyllistä opetettaville \cite{Harper02}. Ohjaajien puutteellinen kokemus tai koulutus opetuksesta voi vaikeuttaa ohjausta tai epäonnistumisen pelkoa ohjaajille \cite{Mark11}. Aiheesta kirjoitetut artikkelit kuitenkin esittävät, että hyvän ohjaajan ei tarvitse olla opinnoissaan kovin paljoa edellä opetettaviaan. Eri oppilaitosten välillä voi olla eroja siinä, kuinka päteviä tai luotettavia ohjaajat ovat tehtävissään \cite{Reges03}. \par

Uusien opiskelijoiden palkkaaminen ohjaajiksi voi olla taloudellisesti epävarmaa, jos ohjaajat ilmoittavat työtuntinsa viikoittain \cite{Reges03}. Tämän vuoksi on vaikea etukäteen ennustaa tarkkaan, kuinka paljon ohjaajien käyttäminen tulee maksamaan, jolloin päättäjien vakuuttaminen ohjaajien käyttämisen eduista voi olla vaikeaa. \par

Lakisääteiset syyt voivat rajoittaa opiskelijoiden hyödyntämistä opetuksessa. Säädökset voivat esimerkiksi kieltää liian pienet kurssit. Vaikka jollakin tietojenkäsittelytieteen alkeiskurssilla olisi valtavat määrät opiskelijoita, voi kurssi tilastollisesti näyttää monelta pieneltä, jos opiskelijat on jaettu pieniin ryhmiin \cite{Reges03}. Säädökset voivat myös kieltää opiskelijoita arvostelemasta saman asteen opiskelijoiden tehtäviä sen varmistamiseksi, että yliopistot eivät laiminlöisi velvollisuuksiaan arvostelussa. Tätä sääntöä voi kiertää esimerkiksi arvostelemalla tehtävät laadullisesti siltä kannalta, kuinka ohjelma yleisesti ottaen toimii eikä keskittymällä koodin yksityiskohtiin \cite{Dickson11}. Tehtäviä voi tarkistaa myös automaattisesti tietokoneen avulla \cite{Vihavainen}. \par

Ohjaajien palkitseminen opintopisteillä palkan sijaan on jakanut mielipiteitä. Vaikka muillakin tieteenaloilla opiskelijoita palkitaan opintopisteillä tutkimusprojekteihin osallistumisesta, on esitetty, että opettamiseen osallistuminen ei olisi yhtä arvokasta toimintaa kuin tutkimukseen osallistuminen \cite{Reges88}. \par

Vaikka ohjaajat pääasiassa hyötyvät siitä, että he ovat suurin piirtein saman ikäisiä kuin opetettavansa, voi siitä seurata myös ongelmia tehtävien arvostelun suhteen \cite{Roberts95}. Koska ohjaajalla ei mahdollisesti ole paljon kokemusta tai auktoriteettia, voi hänen olla vaikea käsitellä opiskelijoita, jotka ovat mielestään saaneet liian huonon arvosanan. Lisäksi uusien ohjaajien voi olla vaikea arvioida, minkä arvosanan antaa epätavalliselle ratkaisulle. Ohjaajilla pitäisikin olla selvät ohjeet tehtävien pisteyttämisen suhteen omaperäisten ratkaisujen tai kiistatilanteiden varalta. Toisaalta jos pidetään tärkeänä sitä, että opiskelijat mieltävät ohjaajat vertaisekseen, voi olla parempi, että ohjaajat eivät osallistu tehtävien pisteyttämiseen ja opiskelijoiden arvosteluun \cite{Morgan02}. Jotkut opiskelijat voivat pitää epämiellyttävänä sitä, että heitä opettaa heitä itseään huomattavasti nuorempi ohjaaja \cite{Sperry08}. \par

Jatkuva vaihtelu opettajan ja opiskelijan roolin välillä voi olla rasittavaa ohjaajalle \cite{Mark11}. Ulkoa opettelua korostavaan opettajakeskeiseen opetukseen tottuneen ohjaajan voi olla vaikeaa siirtyä opettajan ja opiskelijan vuorovaikutusta korostavaan opetustyyliin. \par

Koulutuksen ja kokemuksen puutteen takia ympäristön häiriöt voivat haitata ohjaajan työskentelyä kokenutta opettajaa enemmän \cite{Patitsas12_3}. Esimerkiksi luokkahuoneen valaistus, pöytien järjestely ja opetustilanteen kesto vaikuttavat ohjaajan työhön. \par

Aina ohjaajille ei ole tarjolla tarvittavaa koulutusta \cite{Shannon98}. Syitä koulutuksen puutteeseen voivat olla rahan tai henkilökunnan puute tai ajatus, että ohjaajat eivät tarvitse koulutusta. Henkilökunnalta voi puuttua kiinnostusta tai taitoa avustaa ohjaajia. \par

Jos tutkimusavustajana toimimista pidetään arvokkaampana toimintana kuin opettamista, opiskelijat voivat olla kiinnostuneempia tutkimukseen pyrkimisestä kuin ohjaajana toimimisesta \cite{Shannon98}. Usein tutkimusavustajia rahoitetaan ohjaajia paremmin ja heillä on enemmän tilaisuuksia työskennellä yliopiston henkilökunnan kanssa. Uhkana voi olla, että parhaat opiskelijat päätyvät tutkimusavustajiksi opettamisen sijaan. \par


\section{Yhteenveto}
Tässä artikkelissa käsittelimme opiskelijoiden hyödyntämistä tie\-to\-jen\-kä\-sit\-te\-ly\-tie\-teen opetuksessa. Opiskelijoiden käyttö opetuksessa mahdollistaa suurempien opiskelijamäärien opettamisen ilman, että kustannukset nousevat tai että olisi turvauduttava passiivisiin massaluentoihin. Ohjaajien käyttö tarjoaa mahdollisuuden antaa opiskelijoille aiempaa enemmän yksilöllistä palautetta ja neuvontaa, mikä nostaa opetuksen laatua. Ohjaajat työskentelevät lähellä opiskelijoita, joten he pystyvät antamaan luennoijalle palautetta kurssin etenemisestä. Ohjaajana toimiminen antaa opiskelijalle hyödyllisiä taitoja ja kokemusta, ja ohjaajien läsnäolo kursseilla rentouttaa kurssin ilmapiiriä ja tarjoaa opiskelijoille helposti lähestyttäviä roolimalleja. Opiskelijoiden hyödyntämisen opetuksessa on havaittu toimivan niin suurissa \cite{Reges03} kuin pienissäkin \cite{Dickson11} oppilaitoksissa. \par

Vaikka opiskelijoita on hyödynnetty tietojenkäsittelytieteen opetuksessa jo pitkään, aihetta on tutkittu niukasti. Opiskelijoiden käyttämistä ohjaajina käsittelevät artikkelit antavat usein yksipuolisen myönteisen kuvan aiheesta, koska mahdollisista haitoista tai huonosti toimivan toteutuksen riskeistä on kirjoitettu vain vähän.  \par

\bibliographystyle{babplain}
\bibliography{lahteet}


\end{document}
