\documentclass[finnish]{tktltiki2}


\usepackage[utf8]{inputenc}
\usepackage{lmodern}
\usepackage{microtype}
\usepackage{amsfonts,amsmath,amssymb,amsthm,booktabs,color,enumitem,graphicx}
\usepackage[pdftex,hidelinks]{hyperref}

\makeatletter
\AtBeginDocument{\hypersetup{pdftitle = {\@title}, pdfauthor = {\@author}}}
\makeatother

\usepackage[fixlanguage]{babelbib}
\selectbiblanguage{finnish}

\usepackage[nottoc,numbib]{tocbibind}
\settocbibname{Lähteet}

\newtheorem{lau}{Lause}
\newtheorem{lem}[lau]{Lemma}
\newtheorem{kor}[lau]{Korollaari}

\theoremstyle{definition}
\newtheorem{maar}[lau]{Määritelmä}
\newtheorem{ong}{Ongelma}
\newtheorem{alg}[lau]{Algoritmi}
\newtheorem{esim}[lau]{Esimerkki}

\theoremstyle{remark}
\newtheorem*{huom}{Huomautus}

\title{Opiskelijoiden käyttö tietojenkäsittelytieteen opettamisessa}
\author{Hanna Arpiainen}
\date{\today}
\level{Aine}
\abstract{Aine.}

\keywords{avainsana 1, avainsana 2, avainsana 3}
\classification{}

\begin{document}

\maketitle
\makeabstract

\tableofcontents
\newpage



\section{Johdanto}
Tietojenkäsittelytieteen opetuksessa tarkoituksena on pystyä antamaan mahdollisimman paljon ja mahdollisimman laadukasta opetusta etenkin alkeiskursseille ilman, että kustannukset kohoavat. Eräs ratkaisu tähän on käyttää opiskelijoita opetuksessa.

\section{Opiskelijoiden käyttäminen opetuksessa}

\section{Ohjaajien roolit}

\section{Opiskelijoiden käyttäminen opetuksessa erilaisissa oppilaitoksissa}
-suuret vs pienet
\\
-yksityinen vs julkinen

\section{Opiskelijoiden käyttämisen edut}
-Taloudelliset, opetuksen laatu, ohjaajien saama kokemus jne

\section{Ongelmat}


\bibliographystyle{babplain}
\bibliography{lahteet}


\end{document}
