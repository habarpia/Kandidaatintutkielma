\documentclass[finnish]{tktltiki2}


\usepackage[utf8]{inputenc}
\usepackage{lmodern}
\usepackage{microtype}
\usepackage{amsfonts,amsmath,amssymb,amsthm,booktabs,color,enumitem,graphicx}
\usepackage[pdftex,hidelinks]{hyperref}

\makeatletter
\AtBeginDocument{\hypersetup{pdftitle = {\@title}, pdfauthor = {\@author}}}
\makeatother

\usepackage[fixlanguage]{babelbib}
\selectbiblanguage{finnish}

\usepackage[nottoc,numbib]{tocbibind}
\settocbibname{Lähteet}

\newtheorem{lau}{Lause}
\newtheorem{lem}[lau]{Lemma}
\newtheorem{kor}[lau]{Korollaari}

\theoremstyle{definition}
\newtheorem{maar}[lau]{Määritelmä}
\newtheorem{ong}{Ongelma}
\newtheorem{alg}[lau]{Algoritmi}
\newtheorem{esim}[lau]{Esimerkki}

\theoremstyle{remark}
\newtheorem*{huom}{Huomautus}

\title{Opiskelijoiden hyödyntäminen tietojenkäsittelytieteen opettamisessa}
\author{Hanna Arpiainen}
\date{\today}
\level{Aine}
\abstract{Aine.}

\keywords{avainsana 1, avainsana 2, avainsana 3}
\classification{}

\begin{document}

\maketitle
\makeabstract

\tableofcontents
\newpage



\section{Johdanto}
Tietojenkäsittelytieteen opetuksessa tarkoituksena on pystyä antamaan mahdollisimman paljon ja mahdollisimman laadukasta opetusta etenkin alkeiskursseille ilman, että kustannukset kohoavat. Eräs ratkaisu tähän on käyttää opiskelijoita opetuksessa.



\section{Tausta}
Tietojenkäsittelytieteen opiskelijoiden lukumäärän vaihtelu on syklistä.\cite{Roberts11} Esimerkiksi kotitietokoneiden ja World Wide Webin yleistyminen kasvattivat alan suosiota ja opiskelijoiden määrä kasvoi, kun taas IT-kuplan puhkeaminen 2000-luvun alussa vähensi opiskelijamäärä. Ajoittaisista taantumisista huolimatta opiskelijoiden määrä on kuitenkin pitkällä ajanjaksolla kasvanut, ja teollisuus tarvitsee yhä enemmän osaajia. Muiden aineiden opiskelijat haluavat myös opiskella tietojenkäsittelytieteiden perusteita, koska he uskovat, että ohjelmointitaidot auttavat heitäkin työmarkkinoilla. Etenkin alkeiskursseille pitäisi siis pystyä järjestämään riittävästi opetusta. 
Usein ongelmana on taloudellisten resurssien puute, joten opiskelupaikkojen lukumäärää voidaan joutua rajoittamaan. Lisäksi etenkin 1980-luvulla osaavista opettajista oli pulaa, joten henkilökuntaan jouduttiin palkkaamaan riittämättömästi koulutettuja ulkopuolisia.
\\
Eräs ratkaisu ongelmaan on kehittää yhteistyötä teollisuuden kanssa, koska sillä on taloudellisia resursseja tukea opetusta. Lisäksi teollisuus tarvitsee osaajia ja kärsii itsekin koulutetun työvoiman puutteesta. Toinen ratkaisu on palkata alkeiskursseille mahdollisimman pätevää henkilökuntaa.
\\
Monet yliopistot ovat päättäneet käyttää opiskelijoita kurssin opetukseen, sillä he ovat halvempia kuin pitkälle koulutettu työvoima. Taloudellisen hyödyn lisäksi opiskelijoiden hyödyntämisen opetuksessa on myös havaittu olevan hyödyllistä monella muullakin tavalla.




\section{Oppilaat opettajina}

\subsection{Ohjaajien roolit}
%Opiskelijat toimivat ohjaajana ryhmälliselle jonkin alkeiskurssin opiskelijoita. Ryhmänohjaajien tehtäviin kuuluu esimerkiksi pitää viikoittaisia keskustelusessioita, joissa käytiin läpi kurssin sisältöä.Keskustelusessiot eivät noudattaneet mitään tiukkaa kaavaa, vaan ryhmänohjaaja muokkasi niiden sisällön vastaamaan ryhmän tarpeita. Ryhmänohjaaja saattoi esimerkiksi kerrata luentojen asioita, esittää lisäesimerkkejä tai vastata opiskelijoiden esittämiin kysymyksiin. Ryhmänohjaajan oli oltava myös joustava; mikäli ryhmä oli ymmärtänyt jonkin tärkeän asian väärin oli ryhmänohjaajan mahdollisesti hylättävä aiemmat suunnitelmansa voidakseen korjata tilanteen.



%Ryhmänohjaajat tapasivat kerran viikossa kaikki ryhmänsä opiskelijat yksilöllisessä palautekeskustelussa. Näissä keskusteluissa ryhmänohjaaja sai kuvan opiskelijan edistymisestä kurssilla. Ryhmänohjaajat myös päivystivät tietokoneluokassa valmiina auttamaan kurssin opiskelijoita kurssiin liittyvissä ongelmissa.

\subsection{Millainen on hyvä ohjaaja?}
On vaikea määritellä tarkkaan, millainen on hyvä ohjaaja, koska niin monet erilaiset ohjaajat ovat onnistuneet tehtävässään. Ohjaajalta vaaditaan kuitenkin aitoa innostusta auttaa muita oppimaan ja vastuullisuutta noudattaa kurssin käytäntöjä. Ohjaajan ei ole välttämätöntä olla ohjelmoinnin erityisosaaja, mutta hänen edellytetään olevan valmis kehittämään itseään pystyäkseen auttamaan opetettaviaan riittävästi.\cite{Reges88} Ohjaaja voi jopa hyötyä siitä, että hänellä on itsellään ollut vaikeuksia oppia ohjelmoimaan, sillä hänen on siten helpompi ymmärtää muiden opiskelijoiden vaikeuksia.\cite{Decker06}
\\
On havaittu, että uudet opiskelijat ovat yleensä parempia ohjaajia kuin vanhemmat, opinnoissaan pidemmälle edenneet opiskelijat\cite{Reges88, Decker06}. Pitkälle edenneiden opiskelijoiden saattaa olla vaikeaa ymmärtää kurssin materiaalia, opetustekniikkaa tai alkeiskurssien opiskelijoiden ongelmia. Ohjattavien voi olla vaikea suhtautua ohjaajaan, jos tämä on heitä kovin paljon vanhempi.



\subsection{Ohjaajien koulutus}

Uudet ohjaajat koulutetaan usein erillisellä kurssilla. 

%Kurssilla voidaan esimerkiksi keskustellaan erilaisista opetustyyleistä, mahdollisesti vaikeiden käsitteiden opettamisesta, tehtävien pisteyttämisestä ja hankalien opiskelijoiden käsittelystä. Kokeneet ohjaajat voivat osallistua kurssille jakaakseen kokemustaan ja päivittääkseen omia taitojaan.

%Joissain yliopistoissa ohjaajat pystyvät vaikuttamaan uusien ohjaajien valintaan esimerkiksi ehdottamalla valintakriteereitä tai valintaprosessin vaiheita.


\subsection{Käytännön organisaatio}





\section{Hyödyt}

\subsection{Yliopistolle}
Opiskelijoiden käyttämisestä ohjaajina on taloudellista etua, sillä heille maksettava palkka on yleensä matalampi kuin kokeneemmille assistenteille, tai heidät palkitaan opintopisteillä.

\subsection{Opetukselle}
Koska opiskelijoiden palkkaaminen ohjaajiksi on halvempaa kuin muiden vaihtoehtojen, voi heitä palkata enemmän. Näin opetettavat saavat enemmän yksilöllistä huomiota.
Mikäli ohjaaja on läsnä esimerkiksi luennolla ja esittää luennotsijalle kysymyksiä aiheesta, rohkaisee se opiskelijoitakin osallistumaan enemmän luennoilla. \cite{Dickson11}

\subsection{Ohjaajille}
Ohjaajat saavat palkkaa tai opintopisteitä, minkä lisäksi opettaminen parantaa heidän ryhmätyö- ja esiintymistaitojaan. Opettaessaan he oppivat itsekin materiaalin paremmin.\cite{Reges03} Ohjaajana toimiminen opettaa vastuukantoa ja liittää opiskelijan paremmin mukaan yliopiston yhteisöön.\cite{Dickson11} Ohjaajat voivat saada toisistaan seuraa ja osallistua keskenään erilaisiin vapaa-ajan aktiviteetteihin.\cite{Roberts95}

%Lisäksi ohjaajat tuntevat kurssin sisällön ja yliopiston tietokoneet paremmin kuin vanhemmat assistentit. 

%Ohjaajien läsnäolo luennoilla on luonut kursseille rennompaa ilmapiiriä. Ohjaajien käyttöönotto on luonut yhteisöllisyyttä sellaisiinkin yliopistoihin, joissa ei aiemmin ollut minkäänlaista kunnollista opiskelijayhteisöä. Naispuolisten ohjaajien käyttäminen on lisännyt naisten määrää tietojenkäsittelytieteen pääaineopiskelijoina.





\section{Ongelmat}
Lakisääteiset syyt voivat rajoittaa opiskelijoiden hyödyntämistä opetuksessa. Lait voivat esimerkiksi kieltää opiskelijoita arvostelemasta saman asteen opiskelijoiden tehtäviä varmistaakseen, että yliopistot eivät laiminlyö velvollisuuksiaan arvostelussa. Lait voivat myös kieltää liian pienet kurssit. Vaikka jollakin tietojenkäsittelytieteen alkeiskurssilla olisi valtavat määrät opiskelijoita, jos heidät on jaettu pieniin ryhmiin, voi yksi iso kurssi tilastollisesti näyttää monelta pieneltä.\cite{Reges03}
\\
Ohjaajien palkitseminen opintopisteillä palkan sijaan on jakanut mielipiteitä. Vaikka muillakin tieteenaloilla opiskelijoita palkitaan opintopisteillä tutkimusprojekteihin osallistumisesta, on esitetetty, että opettamiseen osallistuminen ei olisi yhtä arvokasta kuin tutkimukseen osallistuminen.\cite{Reges88}

\bibliographystyle{babplain}
\bibliography{lahteet}


\end{document}
