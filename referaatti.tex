\documentclass[finnish]{tktltiki2}


\usepackage[utf8]{inputenc}
\usepackage{lmodern}
\usepackage{microtype}
\usepackage{amsfonts,amsmath,amssymb,amsthm,booktabs,color,enumitem,graphicx}
\usepackage[pdftex,hidelinks]{hyperref}

\makeatletter
\AtBeginDocument{\hypersetup{pdftitle = {\@title}, pdfauthor = {\@author}}}
\makeatother

\usepackage[fixlanguage]{babelbib}
\selectbiblanguage{finnish}

\usepackage[nottoc,numbib]{tocbibind}
\settocbibname{Lähteet}

\newtheorem{lau}{Lause}
\newtheorem{lem}[lau]{Lemma}
\newtheorem{kor}[lau]{Korollaari}

\theoremstyle{definition}
\newtheorem{maar}[lau]{Määritelmä}
\newtheorem{ong}{Ongelma}
\newtheorem{alg}[lau]{Algoritmi}
\newtheorem{esim}[lau]{Esimerkki}

\theoremstyle{remark}
\newtheorem*{huom}{Huomautus}

\title{Opiskelijoiden käyttö tietojenkäsittelytieteen opettamisessa Stanfordin yliopistossa}
\author{Hanna Arpiainen}
\date{\today}
\level{Kandidaatintutkielma}
\abstract{Referaatti.}

\keywords{avainsana 1, avainsana 2, avainsana 3}
\classification{}

\begin{document}

\maketitle
\makeabstract

\tableofcontents
\newpage



\section{Johdanto}

Stuart Reges, John McGrory ja Jeff Smith kirjoittavat artikkelissaan ”The effective use of undergraduates to staff large introductory CS courses” (1988) mitä etua Stanfordin yliopistossa saatiin opiskelijoiden käyttämisestä alkeiskurssien opetuksessa. Yliopistoilla oli tavoitteena parantaa tietojenkäsittelytieteen alkeiskurssien opetuksen laatua ja määrää ilman, että kustannukset kohoaisivat. Artikkelin mukaan Stanfordin yliopistossa näissä tavoitteissa onnistuttiin käyttämällä opiskelijoita opettamiseen. Stanfordissa luotiin myös kurssi CS198, jossa opetettiin tietojenkäsittelytieteen opettamista.



\section{Erilaiset kokeilut}

Uusien opiskelijoiden hyödyntämiseen tehokkaasti opetuksessa kokeiltiin useita erilaisia rakenteita. Ensimmäisten kokeilujen ongelmana oli se, että tehtävien arvostelijat eivät tienneet tarpeeksi hyvin, mikä kurssilla oli tärkeää. Lisäksi joidenkin mielestä oli ongelmallista palkita opiskelijat opintopisteillä palkan sijaan. Lopulta päädyttiin malliin, jossa uudet opiskelijat saavat opintopisteitä ja vanhemmat palkkaa.



\section{Opiskelijoiden käyttämisen edut}

- tuntevat opetettavat opiskelijat, materiaalin ja tietokoneet, lähellä opiskelijoita
-tekninen osaaminen?



\section{Opiskelijoiden työtehtävät}

Kukin CS198:n opiskelija toimi ohjaajana ryhmälliselle alkeiskurssin opiskelijoita. Ryhmänohjaajien tehtäviin kuului järjestää ryhmälleen viikoittainen keskustelusessio, jossa käytiin läpi kurssin sisältöä ja esiteltiin esimerkkiratkaisuja. Lisäksi ryhmänohjaajat tapasivat kerran viikossa kaikki ryhmänsä opiskelijat yksilöllisessä palautekeskustelussa. Ryhmänohjaajat myös päivystivät tietokoneluokassa valmiina auttamaan kurssin opiskelijoita.


\section{Opettamisen oppiminen}

Tärkeä osa CS198:a oli opiskelijoiden saama kokemus opettamisesta. Opiskelijat osallistuivat tapaamisiin, joissa keskusteltiin opetustavoista ja simuloitiin opetustilanteita. Kokeneemmat ryhmänohjaajat pystyivät jakamaan kokemustaan uusille. Tapaamissa myös pyrittiin jakamaan tietoa eri kurssien välillä.



\bibliographystyle{babplain}
\bibliography{references-fi}


\end{document}
